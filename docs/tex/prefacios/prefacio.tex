\thispagestyle{empty}

\begin{center}
{\large\bfseries My Many Reads \\ Plataforma social para lectores con sistema de recomendaciones }\\
\end{center}
\begin{center}
Ángel Píñar Rivas\\
\end{center}

%\vspace{0.7cm}

\vspace{0.5cm}
\noindent{\textbf{Palabras clave}: \textit{software libre}, \textit{desarrollo ágil}, \textit{integración continua}, \textit{proceso de desarrollo}, \textit{control de calidad}, \textit{biblioteca}, \textit{gestión}, \textit{lecturas}, \textit{libros}, \textit{recomendaciones}, \textit{estadísticas}, \textit{popularidad}, \textit{tendencias}, \textit{usuarios}
\vspace{0.7cm}

\noindent{\textbf{Resumen}\\

El objetivo de este proyecto es definir un proceso de desarrollo claro y robusto, apoyándose en una mentalidad ágil y un conjunto de herramientas que garanticen la buena calidad del código y su documentación. Para poner estas prácticas en uso, se desarrollará un sistema de gestión de lecturas que ofrezca recomendaciones automáticas basadas en las lecturas de cada usuario además de la posibilidad de participar activamente en la plataforma creando recomendaciones personalizadas y otro contenido que pueda resultar de interés al resto de usuarios. Mediante la información y estadísticas de uso generadas en esta plataforma, se obtendrá el beneficio necesario para cubrir costes de mantenimiento.

\cleardoublepage

\begin{center}
	{\large\bfseries My Many Reads \\ Social platform for readers with recommendation system }\\
\end{center}
\begin{center}
	Ángel Píñar Rivas\\
\end{center}
\vspace{0.5cm}
\noindent{\textbf{Keywords}: \textit{open source}, \textit{agile development}, \textit{continuous integration}, \textit{development process}, \textit{quality assurance}, \textit{library}, \textit{management}, \textit{readings}, \textit{books}, \textit{recommendations}, \textit{statistics}, \textit{popularity}, \textit{trends}, \textit{users}
\vspace{0.7cm}

\noindent{\textbf{Abstract}\\

The main goal of this project is to define a clear and robust development process, relying on an agile mindset and a set of tools to ensure good quality of both code and documentation. To put these practices into use, a readings management system will be developed to offer automatic recommendations based on the readings of each user as well as the possibility to actively participate in the platform by creating personalized recommendations and other content that may be of interest to the rest of users. Through the information and usage statistics generated on this platform, the necessary profit will be obtained to cover maintenance costs.

\cleardoublepage

\thispagestyle{empty}

\noindent\rule[-1ex]{\textwidth}{2pt}\\[4.5ex]

D. \textbf{Juan Julián Merelo Guervós}, Profesor del Departamento de Arquitectura y Tecnología de Computadores de la Universidad de Granada.

\vspace{0.5cm}

\textbf{Informo:}

\vspace{0.5cm}

Que el presente trabajo, titulado \textit{\textbf{My Many Reads}},
ha sido realizado bajo mi supervisión por \textbf{Ángel Píñar Rivas}, y autorizo la defensa de dicho trabajo ante el tribunal
que corresponda.

\vspace{0.5cm}

Y para que conste, expiden y firman el presente informe en Granada a Septiembre de 2022.

\vspace{1cm}

\textbf{El director: }

\vspace{5cm}

\noindent \textbf{(Juan Julián Merelo Guervós)}

\chapter*{Agradecimientos}

A la resistencia, por el apoyo frente a las oleadas incesantes de presión y trabajo durante este máster, y la sensación de compañerismo en tiempos de pandemia. Sin vosotros, la sensación de soledad me habría llevado a abandonar este curso.

A Phoenix, por su empatía, su disposición a escuchar y la sensación de espalda cubierta, todo ello sin obligación de hacerlo.

A mi tutor, por aguantar mis cabezonerías con mi propia forma de trabajo, las revisiones frecuentes y las discusiones fructíferas que me han enseñado tanto información como puntos de vista.

A mis amigos y familia por escuchar y convertir mi agobio y preocupaciones en risas y diversión, cuidando de mi salud mental.

Y finalmente a mí mismo (aunque suene feo), por resistir la tentación de abandonar y seguir adelante hasta el final, aún con todo el agobio y estrés, y las motivaciones desaparecidas.


