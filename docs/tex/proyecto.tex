% \documentclass[paper=a4, fontsize=11pt]{scrartcl} % A4 paper and 11pt font size
\documentclass[11pt, a4paper]{book}
\usepackage[T1]{fontenc} % Use 8-bit encoding that has 256 glyphs
\usepackage[utf8]{inputenc}
\usepackage{fourier} % Use the Adobe Utopia font for the document - comment this line to return to the LaTeX default
\usepackage{listings} % para insertar código con formato similar al editor
\usepackage[spanish, es-tabla]{babel} % Selecciona el español para palabras introducidas automáticamente, p.ej. "septiembre" en la fecha y especifica que se use la palabra Tabla en vez de Cuadro
\usepackage{url} % ,href} %para incluir URLs e hipervínculos dentro del texto (aunque hay que instalar href)
\usepackage{graphics,graphicx, float} %para incluir imágenes y colocarlas
\usepackage[gen]{eurosym} %para incluir el símbolo del euro
\usepackage{cite} %para incluir citas del archivo <nombre>.bib
\usepackage{enumerate}
\usepackage{hyperref}
\usepackage{graphicx}
\usepackage{tabularx}
\usepackage{booktabs}
\usepackage[table,xcdraw]{xcolor}
\usepackage{enumitem}
\usepackage{titlesec}

\setlength{\parskip}{2.2ex}
\titlespacing\section{0pt}{5.5ex plus 1ex minus .2ex}{1.5ex}\relax
\titlespacing\subsection{0pt}{5.5ex plus 1ex minus .2ex}{1.5ex}\relax


\lstset{frame=tb,
  language=Python,
  breaklines=true,
  showstringspaces=false,
  columns=flexible,
  numbers=none,
  commentstyle=\color{dkgreen},
  stringstyle=\color{mauve},
  tabsize=3
}

\usepackage[table,xcdraw]{xcolor}
\hypersetup{
	colorlinks=true,	% false: boxed links; true: colored links
	linkcolor=black,	% color of internal links
	urlcolor=cyan		% color of external links
}

\renewcommand{\familydefault}{\sfdefault}
\usepackage{fancyhdr} % Custom headers and footers
\pagestyle{fancyplain} % Makes all pages in the document conform to the custom headers and footers
\fancyhead[L]{} % Empty left header
\fancyhead[C]{} % Empty center header
\fancyhead[R]{Ángel Píñar Rivas} % My name
\fancyfoot[L]{} % Empty left footer
\fancyfoot[C]{} % Empty center footer
\fancyfoot[R]{\thepage} % Page numbering for right footer
%\renewcommand{\headrulewidth}{0pt} % Remove header underlines
\renewcommand{\footrulewidth}{0pt} % Remove footer underlines
\setlength{\headheight}{13.6pt} % Customize the height of the header

\usepackage{titlesec, blindtext, color}
\definecolor{gray75}{gray}{0.75}
\newcommand{\hsp}{\hspace{20pt}}
\titleformat{\chapter}[hang]{\Huge\bfseries}{\thechapter\hsp\textcolor{gray75}{|}\hsp}{0pt}{\Huge\bfseries}
\setcounter{secnumdepth}{4}
\usepackage[Lenny]{fncychap}


\begin{document}
	\begin{titlepage}
\newlength{\centeroffset}
\setlength{\centeroffset}{-0.5\oddsidemargin}
\addtolength{\centeroffset}{0.5\evensidemargin}
\thispagestyle{empty}

\noindent\hspace*{\centeroffset}\begin{minipage}{\textwidth}

\centering
\includegraphics[width=0.9\textwidth]{logos/logo_ugr.jpg}\\[1.4cm]

\textsc{ \Large TRABAJO FIN DE MÁSTER\\[0.2cm]}
\textsc{ GRADO EN INGENIERÍA INFORMÁTICA}\\[1cm]

{\Huge\bfseries My Many Reads \\}
\noindent\rule[-1ex]{\textwidth}{3pt}\\[3.5ex]
{\large\bfseries Plataforma social para lectores con sistema de recomendaciones }
\end{minipage}

\vspace{2.5cm}
\noindent\hspace*{\centeroffset}
\begin{minipage}{\textwidth}
\centering

\textbf{Autor}\\ {Ángel Píñar Rivas}\\[2.5ex]
\textbf{Director}\\ {Juan Julián Merelo Guervós}\\[2cm]
\includegraphics[width=0.3\textwidth]{logos/etsiit_logo.png}\\[0.1cm]
\textsc{Escuela Técnica Superior de Ingenierías Informática y de Telecomunicación}\\
\textsc{---}\\
Granada, Septiembre de 2022
\end{minipage}
\end{titlepage}

	\thispagestyle{empty}

\begin{center}
{\large\bfseries My Many Reads \\ Plataforma social para lectores con sistema de recomendaciones }\\
\end{center}
\begin{center}
Ángel Píñar Rivas\\
\end{center}

%\vspace{0.7cm}

\vspace{0.5cm}
\noindent{\textbf{Palabras clave}: \textit{software libre}, \textit{desarrollo ágil}, \textit{integración continua}, \textit{proceso de desarrollo}, \textit{control de calidad}, \textit{biblioteca}, \textit{gestión}, \textit{lecturas}, \textit{libros}, \textit{recomendaciones}, \textit{estadísticas}, \textit{popularidad}, \textit{tendencias}, \textit{usuarios}
\vspace{0.7cm}

\noindent{\textbf{Resumen}\\

El objetivo de este proyecto es definir un proceso de desarrollo claro y robusto, apoyándose en una mentalidad ágil y un conjunto de herramientas que garanticen la buena calidad del código y su documentación. Para poner estas prácticas en uso, se desarrollará un sistema de gestión de lecturas que ofrezca recomendaciones automáticas basadas en las lecturas de cada usuario además de la posibilidad de participar activamente en la plataforma creando recomendaciones personalizadas y otro contenido que pueda resultar de interés al resto de usuarios. Mediante la información y estadísticas de uso generadas en esta plataforma, se obtendrá el beneficio necesario para cubrir costes de mantenimiento.

\cleardoublepage

\begin{center}
	{\large\bfseries My Many Reads \\ Social platform for readers with recommendation system }\\
\end{center}
\begin{center}
	Ángel Píñar Rivas\\
\end{center}
\vspace{0.5cm}
\noindent{\textbf{Keywords}: \textit{open source}, \textit{agile development}, \textit{continuous integration}, \textit{development process}, \textit{quality assurance}, \textit{library}, \textit{management}, \textit{readings}, \textit{books}, \textit{recommendations}, \textit{statistics}, \textit{popularity}, \textit{trends}, \textit{users}
\vspace{0.7cm}

\noindent{\textbf{Abstract}\\

The main goal of this project is to define a clear and robust development process, relying on an agile mindset and a set of tools to ensure good quality of both code and documentation. To put these practices into use, a readings management system will be developed to offer automatic recommendations based on the readings of each user as well as the possibility to actively participate in the platform by creating personalized recommendations and other content that may be of interest to the rest of users. Through the information and usage statistics generated on this platform, the necessary profit will be obtained to cover maintenance costs.

\cleardoublepage

\thispagestyle{empty}

\noindent\rule[-1ex]{\textwidth}{2pt}\\[4.5ex]

D. \textbf{Juan Julián Merelo Guervós}, Profesor del Departamento de Arquitectura y Tecnología de Computadores de la Universidad de Granada.

\vspace{0.5cm}

\textbf{Informo:}

\vspace{0.5cm}

Que el presente trabajo, titulado \textit{\textbf{My Many Reads}},
ha sido realizado bajo mi supervisión por \textbf{Ángel Píñar Rivas}, y autorizo la defensa de dicho trabajo ante el tribunal
que corresponda.

\vspace{0.5cm}

Y para que conste, expiden y firman el presente informe en Granada a Septiembre de 2022.

\vspace{1cm}

\textbf{El director: }

\vspace{5cm}

\noindent \textbf{(Juan Julián Merelo Guervós)}

\chapter*{Agradecimientos}

A la resistencia, por el apoyo frente a las oleadas incesantes de presión y trabajo durante este máster, y la sensación de compañerismo en tiempos de pandemia. Sin vosotros, la sensación de soledad me habría llevado a abandonar este curso.

A Phoenix, por su empatía, su disposición a escuchar y la sensación de espalda cubierta, todo ello sin obligación de hacerlo.

A mi tutor, por aguantar mis cabezonerías con mi propia forma de trabajo, las revisiones frecuentes y las discusiones fructíferas que me han enseñado tanto información como puntos de vista.

A mis amigos y familia por escuchar y convertir mi agobio y preocupaciones en risas y diversión, cuidando de mi salud mental.

Y finalmente a mí mismo (aunque suene feo), por resistir la tentación de abandonar y seguir adelante hasta el final, aún con todo el agobio y estrés, y las motivaciones desaparecidas.




	\newpage
	\tableofcontents

	\newpage
	\listoffigures

	\listoftables 
	\newpage

	\chapter{Introducción}

A día de hoy, el afán por la lectura está aumentando entre no sólo los jóvenes (el sector que más preocupaba) sino adultos también. Una de las posibles causas de este hecho podría haber sido la pandemia, ya que con la cuarentena muchos han tenido que encontrar nuevos \textit{hobbies} a los que dedicarse desde su hogar para reemplazar actividades al aire libre.

Como con cualquier otro tipo de afición, no es sencillo iniciarse en el mundo de la lectura sin conocer un buen punto de partida, pues lo ideal es comenzar con algo fácilmente disfrutable, lo cual depende fuertemente de los gustos personales de la persona en concreto, por lo que en general se suele hacer una pequeña investigación sobre aquellos que podrían ser interesantes.

Este tipo de búsquedas no solo las realizan aquellas personas introduciéndose en la lectura, sino también los ya acostumbrados lectores que acaban con su lista de libros pendientes o prefieren descubrir libros nuevos. Al fin y al cabo, cualquier persona interesada en la lectura acaba buscando algo que leer por los medios de que disponga.

\section{Motivación}

Justo antes de la pandemia decidí explorar nuevos \textit{hobbies} que me separaran de la pantalla. Compré varios juegos de mesa, empecé a pintar miniaturas y decidí comprarme un libro con una premisa medianamente interesante. De estos tres, las miniaturas fueron las únicas que acabé descartando, porque aunque me gustan, son bastante caras, y como para los juegos de mesa dependes de organizarte con tus amigos, me enfoqué bastante en la lectura.

Me encontré con el problema de que no conocía a mucha gente que leyera, y además, que compartiera mis gustos, por lo que dependía de buscar en internet en base a lo poco que llevaba leído, y la verdad es que era una tarea algo tediosa. Se me ocurrió que podría existir una plataforma de \textit{reviews} de la que podría servirme, y me topé con \textit{Goodreads}, pero la verdad es que el sitio no terminaba de gustarme del todo. Conozco otras plataformas para series y películas que sí son bastante cómodas de usar para estas búsquedas, pero no para libros.

Finalmente me acabé acostumbrando bastante a mi método ``manual`` de buscar nuevos libros, introduje a un par de amigos con mis mismos gustos a la lectura para discutir con ellos y acabé disfrutando más del \textit{hobby} en general. Sin embargo, siempre pensé (y sigo pensando) que todo esto habría sido más fácil e incluso mejor, si hubiera una plataforma online en la que explorar libros, compartir opiniones y recomendaciones, y gestionar las lecturas, en cierto modo parecido a lo que ofrece \textit{Goodreads} pero eliminando todos los detalles, aspectos y funcionalidades que hacen que yo, a pesar de disfrutar mucho de la lectura y de llevarme bien con los ordenadores en general, no quiera usar esta plataforma que parece estar pensada para gente con mi misma afición, y que además es la más destacable de este ámbito con diferencia.

\section{Descripción del problema}

Existe una gran diversidad de géneros y gustos personales a tener en cuenta a la hora de realizar una buena elección de un libro. En el caso de no tener contactos que puedan aconsejar, habría que recurrir a otras fuentes de información, principalmente internet ya que es la más accesible, amplia y fácil de utilizar de las disponibles.

Se pueden encontrar ideas o sugerencias con simples búsquedas a través de Google, pero éstas suelen venir de revistas o blogs cuyos motivos de dichas recomendaciones no son transparentes, pudiendo ser, por ejemplo, totalmente personales del autor (una única persona) o haberse visto influenciados por un movimiento de patrocinio por parte de alguna editorial.

Esto no parece una solución ideal, pues un usuario promedio fácilmente puede desconfiar de estas recomendaciones, ya que por una parte se desconoce de dónde proceden y cuales son los motivos detrás de éstas, y por otra parte podrían no encajar con los gustos del usuario.

El propósito de este proyecto es acabar con este problema, permitiendo a todo aquel usuario con interés en la lectura servirse de unas recomendaciones totalmente personalizadas, apoyándose por opiniones que una multitud de usuarios similares aportan para cada libro. En otras palabras, elaborando las recomendaciones para los usuarios en base a sus lecturas y respaldando esas recomendaciones con opiniones de otros usuarios reales, por lo que se acaba creando una plataforma social en torno a la lectura en la que se refleja la opinión mayoritaria de sus usuarios, aportando mucha fiabilidad.

Por otro lado, también existen editoriales y anunciantes que deben tomar las decisiones correctas a la hora de decidir que libros publicar para obtener buenos beneficios, así como campañas publicitarias. Con el fin de hacer el mejor movimiento posible, estas entidades se apoyan en estudios de mercado. Éstos recogen información sobre los potenciales clientes, ayudando a enfocar la publicidad a un sector concreto y realizando una estimación de los ingresos que podrían obtenerse con la venta del libro.

Recabar toda esta información puede ser un proceso muy complicado y tedioso. Para ello hay que buscar diferentes fuentes, teniendo en cuenta que cada una probablemente emplee un formato distinto. Una vez determinadas se  recopila y filtra la información extraída de éstas para finalmente obtener solo aquellos datos útiles y de interés a partir de los cuales se conforma el estudio de mercado.

Un cliente lector y un cliente dedicado a la venta de libros, podrían encontrar la solución a los problemas mencionados anteriormente en un mismo sitio. Ésto es lo que se pretende lograr con \textit{My Many Reads}.

\section{Objetivos}
En el caso de \textit{My Many Reads}, la meta global es construir una plataforma social en la que crear una comunidad de usuarios alrededor de la lectura y aprovechar el tráfico de información generado para obtener beneficios a partir de terceros. Para alcanzar esta meta, se define una lista de objetivos con el fin de planificar y diseñar en torno al cumplimiento de estos, enfocando todo el desarrollo en una dirección:

\begin{enumerate}[label=\textbf{O-\arabic*}]
    \item Cumplir con una funcionalidad mínima de consulta de libros y gestión listas o bibliotecas.
    \item Con el fin de que los usuarios puedan encontrar fácilmente buenas sugerencias para sus próximas lecturas, construir un sistema de recomendaciones que, a partir de los gustos del usuario, devuelva una lista de libros que podrían ser de interés para éste.
    \item Permitir a los usuarios participar activamente en creación de contenido de la plataforma a través de recomendaciones entre libros personalizadas, \textit{reviews} y publicaciones en foros de discusiones.
    \item Servir estadísticas de los usuarios de la plataforma para conocer información sobre, por ejemplo, las tendencias actuales.
\end{enumerate}

Estos dos últimos objetivos tienen cierta dependencia a que el proyecto se encuentre desplegado y con una base mínima de usuarios. Para llegar a ese punto, es imprescindible cumplir los objetivos previos cuya intención es atraer usuarios a la plataforma, y lograr cierto nivel de éxito.

Una vez definida esta lista de objetivos a alcanzar, el siguiente paso es determinar una serie de funcionalidades a través de las cuales se cumplimentarán estos objetivos:

\begin{enumerate}[label=\textbf{F-\arabic*}]
    \item Gestión de una biblioteca personal, en la que los usuarios puedan almacenar los libros leídos, pendientes y abandonados junto a una valoración.
    \item Sistema de recomendaciones automático construyendo los gustos de los usuarios a partir de sus bibliotecas personales.
    \item Acceso a foros de discusiones para cada libro en el que plasmar opiniones, teorías u otros pensamientos relacionados con dicho libro.
    \item Posibilidad de escribir recomendaciones personalizadas entre dos libros similares, que puedan ayudar a otros usuarios a tomar una decisión con su próxima lectura.
    \item Sistema de búsqueda de libros por popularidad en la plataforma o sus características propias como pueden ser géneros, idioma o fecha de publicación.
    \item Recolección y publicación de estadísticas indicativas de la popularidad y tendencias actuales en el mundo de la lectura.
\end{enumerate}

\section{\textit{User journeys} como ejemplos}
\label{user journeys}

En esta sección se pretende demostrar cómo, a través del cumplimiento de los objetivos enumerados anteriormente, se ofrece una solución al problema descrito que además pueda dejar una sensación de satisfacción a los usuarios. A través de un \textit{user journey} (poco detallado, pues aún no se ha definido completamente \textit{My Many Reads}) se puede ejemplificar el problema desde diversas perspectivas, además de justificar la definición de estos objetivos como solución a éste.

\textit{Sam} disfruta leyendo tanto libros en edición física como digital, ya que algunos solo se encuentran disponible en una versión o prefiere llevarlos consigo más cómodamente. Para organizarlos en un lugar común y además gestionarlos de manera sencilla, recurre a una biblioteca virtual. A continuación, veamos el proceso básico que un usuario, en este caso \textit{Sam}, habría de seguir para crear una biblioteca de libros de una temática concreta.

\begin{itemize}
    \item \textbf{Acciones}
    \begin{itemize}
        \item Crear nueva biblioteca.
        \begin {itemize}
            \item Nombrarla ``Crecimiento personal``.
        \end{itemize}
        \item Para cada libro pendiente de añadir:
        \begin{itemize}
            \item Buscar su página correspondiente.
            \item Añadir a la biblioteca recién creada junto con otras posibles anotaciones: estado (pendiente, completado, en lectura), calificación, comentarios personales...
        \end{itemize}
    \end{itemize}
    \item \textbf{Puntos de contacto}
    \begin{itemize}
        \item Biblioteca de libros del usuario.
        \item Página de búsqueda de libros.
        \item Página de un libro.
    \end{itemize}
    \item \textbf{Pensamientos que debería tener}
    \begin{itemize}
        \item El proceso es intuitivo.
        \item La biblioteca queda claramente organizada, y se puede consultar de manera sencilla.
    \end{itemize}
\end{itemize}

Con estas bibliotecas, \textit{Sam} ya puede tener sus lecturas organizadas en un entorno virtual y gestionable, cómodo y sencillo de gestionar. Pero ahora \textit{Sam} busca algo nuevo que leer, pues su lectura actual está a punto de acabar, por lo que le gustaría recibir recomendaciones basadas en sus gustos. Para ello:

\begin{itemize}
    \item \textbf{Acciones}
    \begin{itemize}
        \item Abrir una de las bibliotecas.
        \item Acceder al apartado de recomendaciones, que generará una lista en base a la biblioteca seleccionada.
        \begin{itemize}
            \item Consultar las ocurrencias en dicho apartado.
            \item Acceder a las páginas de aquellos libros que resulten más interesantes para obtener más información.
        \end{itemize}
    \end{itemize}
    \item \textbf{Puntos de contacto}
    \begin{itemize}
        \item Biblioteca de libros del usuario.
        \item Lista de recomendaciones de una biblioteca.
        \item Página de un libro.
    \end{itemize}
    \item \textbf{Pensamientos que debería tener}
    \begin{itemize}
        \item Las recomendaciones son apropiadas, van de acuerdo a las lecturas completadas de la biblioteca escogida, y causan interés.
        \item Volvería a consultar este apartado en caso de querer leer algo nuevo.
    \end{itemize}
\end{itemize}

A través de estos \textit{user journeys}, aunque simplificados y con bajo nivel de detalle, queda demostrado como unos usuarios que experimentan unos problemas concretos pueden beneficiarse de un sistema que cumpla los objetivos listados en la sección anterior para solventar total o parcialmente sus dificultades.

Conforme avance este proyecto, se aumentará el nivel de detalle y, por tanto, incrementará la complejidad del sistema propuesto, pero siempre con estos objetivos a cumplir como meta.
	\chapter{Estado del arte}
\label{Estado del arte}

Con lo descrito hasta ahora, se pueden identificar fácilmente dos tipos de usuarios o clientes de \textit{My Many Reads}, aunque puede que en el futuro aparezcan más. Estos son:
\begin{itemize}
    \item El \textbf{usuario lector}, que visita la plataforma como un punto de interés y apoyo para su \textit{hobby}. \label{usuario lector}
    \item El \textbf{usuario editorial}, que consume el tráfico de información generado con el fin de introducir anuncios o hacer estudios de mercado.
\end{itemize}

\section{Problemas de los usuarios}
\label{problemas de los usuarios}

Entre estos dos usuarios se puede generar la siguiente lista de problemas:
\begin{itemize}
    \item \textbf{Usuario lector}
    \begin{itemize}
        \item Necesita una herramienta para gestionar mis lecturas.
        \item Necesita recomendaciones de libros nuevos, que se adapten a mis gustos.
        \item Necesita explorar libremente los libros existentes por características como la popularidad, en caso de querer explorar nuevas opciones.
        \item Necesita conocer libros similares entre sí, junto a los motivos de la similitud, en caso de querer leer algo parecido a un libro que ya ha leído.
    \end{itemize}
    \item \textbf{Usuario editorial}
    \begin{itemize}
        \item Necesita saber cuales son las tendencias más populares, como autores o géneros, para conocer el estado del mercado actual.
        \item Necesita publicitar libros en lugares concurridos por la audiencia adecuada.
    \end{itemize}
\end{itemize}

Actualmente solo destacan otras dos plataformas similares, Goodreads y Librarything, siendo la primera la opción más conocida y popular en cuanto a este tipo de plataformas. A continuación se exploran las características de estas dos plataformas, listando aspectos positivos y negativos.

\section{Plataformas existentes}

En esta sección se exploran las características de algunas plataformas ya existentes similares a \textit{My Many Reads}, resolviendo problemas comunes y compartiendo objetivos.

En sus correspondientes secciones se realiza un análisis exploratorio de ambas plataformas, enumerando aspectos positivos y negativos con el fin de descubrir posibles fortalezas y debilidades a tener en cuenta durante el desarrollo de \textit{My Many Reads}, y finalmente incluyendo una tabla en la que se indica qué problemas de los mencionados anteriormente solventa y hasta que punto.

\subsection[Goodreads]{Goodreads\\ {\large \url{https://www.goodreads.com/}}}

Sin duda la opción más popular, esta plataforma ofrece una serie de opciones bastante llamativas y útiles:

\begin{itemize}
    \item Una biblioteca de libros a la que puedes añadir libros leídos, en progreso y en lista de espera, junto con tu puntuación, fechas de inicio y fin de lectura y \textit{review}.
    \item Permite añadir libros por tu cuenta, en caso de que estos no se encuentren en la web de la página por ser ediciones antiguas, editoriales pequeñas o cualquier otro motivo.
    \item Un sistema de recomendaciones basado en tu biblioteca.
    \item Estadísticas de tus lecturas.
    \item Listas creadas por la comunidad que recogen libros con características comunes, como puede ser \textit{Books That Should Be Made Into Movies}.
    \item Un foro de discusiones, un apartado de citas famosas y secciones de AMA (\textit{Ask Me Anything}) dirigidas a autores.
    \item Una sección de noticias y artículos relacionados con los libros y el mundo de la lectura en general.
    \item Elementos de \textit{gamificación} de la lectura como preguntas diarias, un \textit{challenge} de lecturas anuales y cuestionarios acerca de libros.
    \item Permite a los usuarios escribir y compartir sus propias historias.
\end{itemize}

No por nada esta plataforma es la más popular, pero sin embargo no carece de aspectos mejorables:

\begin{itemize}
    \item La interfaz en sí es bastante mejorable, se hace difícil a un usuario promedio navegar y descubrir todo lo que ofrece sin tener que realizar un poco de esfuerzo.
    \item No permite explorar cómodamente la base de datos de libros, solo puedes ver las tendencias actuales (decididas por la plataforma mediante algún algoritmo desconocido) y explorar de un género concreto.
    \item Las listas mencionadas anteriormente las crea un único usuario, por lo que solo la opinión de una persona es la que dicta la relación entre todos los libros de la lista.
    \item El apartado de discusión de libros no ofrece mucha libertad, una vez lo encuentras parece más bien un conjunto de publicaciones filtrado según si son acerca de o relacionados con el libro concreto.
    \item Al visitar la página de un libro, si quieres consultar opiniones de la gente, lo que se te ofrece es su puntación media (no precisamente muy visible) y una sección de comentarios como la que te podrías encontrar en Amazon o cualquier tienda online. El foro de discusiones mencionado anteriormente es una fuente valiosa de \textit{feedback} que, aún con los problemas mencionados anteriormente, no parece ser accesible desde esta página.
    \item Información adicional como la existencia de adaptaciones en otros formatos (tales como películas o series) no aparece por ninguna parte.
    \item No se hace mención a diferentes ediciones de un libro ni a los idiomas en los que está disponible.
    \item El autor, otro elemento del cual un usuario podría usar para guiarse a la hora de buscar su siguiente lectura, solo dispone de un perfil muy similar al que podría tener cualquier usuario común, diferenciándose en una sección de AMA y de listado de sus libros. No parece haber un lugar en el que discutir sobre el estilo narrativo u otros elementos comunes a todos sus libros, que puedan llevar a un usuario a leer diferentes libros del autor que no compartan género entre sí.
\end{itemize}

En la tabla siguiente se muestra el grado de cumplimiento que tiene esta plataforma con respecto a los problemas listados anteriormente:

\begin{table}[H]
    \begin{tabularx}{\linewidth}{|X|X|}
        \hline
        \textbf{Problema} & \textbf{¿Lo solventa \textit{Goodreads}?} \\
        \hline
        \hline
        \multicolumn{2}{|l|}{\textbf{Usuario lector}} \\
        \hline
        Tiene una herramienta para gestionar lecturas     & \textbf{Sí}, con \textit{shelves} para categorizar tus libros \\ 
        \hline
        Ofrece recomendaciones adaptadas a los gustos del usuario & \textbf{Sí}, según géneros, \textit{shelves} y privadas de otros usuarios \\ 
        \hline
        Permite explorar libros estableciendo criterios de búsqueda   & \textbf{Parcialmente}, sólo por géneros \\ 
        \hline
        Permite consultar libros relacionados con otros   & \textbf{Parcialmente}, pero sin justificación \\ 
        \hline
        \multicolumn{2}{|l|}{ \textbf{Usuario editorial}} \\
        \hline
        Muestra las tendencias actuales     & \textbf{Parcialmente}, muestra 25 tendencias entre todos los usuarios del sitio, además de unos premios anuales. No permite por ejemplo tendencias por género o autor. \\ 
        \hline
        Muestra publicidad dirigida     & \textbf{Sí}, a través de artículos, solo uno por libro. Para ver más hay que ir a una sección propia. Además, muestra enlaces de compra en tiendas online. \\ 
        \hline
    \end{tabularx}
\end{table}

\subsubsection{Modelo de negocio}

En esta plataforma, el modelo de negocio queda claramente visible. En el pie de página se puede observar una sección denominada \textit{Work with us} en la que aparecen enlaces de interés para autores y anunciantes que puedan estar interesados en darse a conocer o adquirir más público a través de la plataforma.

Por otra parte, y no quedando reflejado tan claro de cara al usuario, también venden información con el objetivo de mostrar publicidad más relevante a los usuarios, y los enlaces a tiendas online para comprar libros que muestra suelen tener incrustados algún código de referencia mediante el cual, probablemente, la página gane cierta comisión por las ventas realizadas a través de dicho enlace.

\subsection[Librarything]{Librarything\\ {\large \url{https://www.librarything.com/}}}

La opción menos popular, no da indicios de la funcionalidad al completo que ofrece hasta que el usuario se registra. Además de ofrecer menos funcionalidad que Goodreads, parece que tiene algunos problemas adicionales. Comenzando por lo que ofrece esta plataforma:

\begin{itemize}
\item Una biblioteca de libros para cada usuario, junto con un sistema de recomendaciones automáticas en base a esta biblioteca.
\item Una sección de comunidades, similares a grupos de lectura, en las que compartir opiniones, embarcarse en retos o hacer concursos.
\item Un foro de discusiones, similar al de Goodreads, pero más accesible.
\item Búsqueda de autores.
\item Conexión de tu perfil con otras aplicaciones, como redes sociales o plataformas de intercambio de libros.
\item Muestra más metadatos sobre los libros que Goodreads, como idiomas en los que está disponibles y portadas que puedes encontrar.
\item Permite consultar estadísticas de los usuarios de la plataforma, lo cual resulta de utilidad a la hora de decidir si involucrarse con esta plataforma.
\end{itemize}

La funcionalidad ofrecida es claramente menor, aunque cabe decir que tiene lo suficiente como para poder atraer a un conjunto de usuarios bastante amplio. Pero de nuevo, no todo se ofrece de la mejor forma posible, existen aspectos susceptibles de mejora:

\begin{itemize}
\item De nuevo, la interfaz resulta claramente mejorable, lo que puede causar pérdida de usuarios.
\item Se usan diferentes bibliotecas según el estado de lectura de un libro, y solo permite al usuario añadir una puntuación y etiquetas a cada libro añadido a la biblioteca.
\item No existe un buscador libros o base de datos propia, lo que permite agregar dos veces el mismo libro si se busca en páginas o ediciones diferentes, por lo que aparecen inconsistencias.
\item Tiene un buscador de librerías o bibliotecas cercanas a tu zona, lo cual no está mal, pero existen servicios similares más populares y fiables, como el propio Google Maps.
\item Como en Goodreads, solo permite explorar libros por género o etiqueta, sin ningún tipo de filtro más allá de eso.
\item No existen listas de libros y las recomendaciones escritas por usuarios no tienen por qué incluir el motivo de la recomendación, lo cual no genera mucha confianza.
\end{itemize}

En la tabla siguiente se muestra el grado de cumplimiento que tiene esta plataforma con respecto a los problemas listados anteriormente:

\begin{table}[H]
    \begin{tabularx}{\linewidth}{|X|X|}
        \hline
        \textbf{Problema} & \textbf{¿Lo solventa \textit{Goodreads}?} \\
        \hline
        \hline
        \multicolumn{2}{|l|}{\textbf{Usuario lector}} \\
        \hline
        Tiene una herramienta para gestionar lecturas     & \textbf{Sí}, con colecciones para categorizar tus libros \\ 
        \hline
        Ofrece recomendaciones adaptadas a los gustos del usuario & \textbf{Sí}, de entre los más populares y los más recientes \\ 
        \hline
        Permite explorar libros estableciendo criterios de búsqueda   & \textbf{Parcialmente}, sólo por términos de popularidad \\ 
        \hline
        Permite consultar libros relacionados con otros   & \textbf{Parcialmente}, por autor, géneros, comunes a los usuarios, pero sin justificación \\ 
        \hline
        \multicolumn{2}{|l|}{ \textbf{Usuario editorial}} \\
        \hline
        Muestra las tendencias actuales     & \textbf{Sí}, en intervalos de tiempo configurables, así como otras estadísticas sobre autores, series e idiomas. \\ 
        \hline
        Muestra publicidad dirigida     & \textbf{Parcialmente}, solo enlaces a tiendas online. \\ 
        \hline
    \end{tabularx}
\end{table}

\subsubsection{Modelo de negocio}

Hasta hace poco, parece ser que esta plataforma subsistía a base de suscripciones mensuales, hasta que debido a la pandemia del Covid-19, decidieron cambiar este aspecto y volverse gratuito.

A pesar de todo esto, siguen sin mostrar anuncios a los usuarios miembro, y su modelo de negocio no termina de estar totalmente claro. Es cierto que ofrecen estadísticas que podrían ser útiles, pero son de acceso libre. 

Teniendo en cuenta este último cambio, y el poco énfasis realizado en mostrar anuncios y publicitar nuevo contenido del mundo de la lectura, no aporta una sensación de tener algún tipo de beneficios desde el punto de vista del usuario promedio, pero en los términos de servicio se indica que esta plataforma se reserva los derechos de vender toda la información anónima, es decir, que no se pueda asociar a una persona concreta.
 
\subsection{Conclusiones}

Aunque también hay que mencionar que existen otras plataformas de este estilo pero con temática diferente, en cuanto a libros hay pocas opciones, y la que parece la mejor de ellas, Goodreads, tiene una amplitud de aspectos que se podrían mejorar, y por tanto introducirse en este entorno no parece una lucha imposible.

A continuación se muestran dos listas, una para enumerar todas las características que me han parecido muy buenas y que se encuentran presentes en alguna de estas plataformas, y otra para todos los aspectos negativos que considero que serían innecesarios o incluso perjudiciales para \textit{My Many Reads}.

Características más destacables:

\begin{itemize}
\item La posibilidad de crear una o varias colecciones (bibliotecas) de libros para recoger tus lecturas, con la posibilidad de añadir anotaciones para indicar el estado de la lectura u otras valoraciones.
\item Un sistema de recomendaciones automático en función de los libros incluidos en tu biblioteca.
\item Mantener una buena relación entre un libro y otros componentes relacionados con este, como información adicional (idiomas disponibles, página del autor, fecha de publicación, ediciones...) o contenido generado por otros usuarios (recomendaciones personalizadas, \textit{reviews}, puntuación...)
\item Proveer de herramientas a los usuarios para que su feedback (opiniones, puntuaciones, recomendaciones, etc.) quede reflejado y sea de utilidad para el resto de usuarios y los algoritmos de la plataforma.
\end{itemize}

Aspectos negativos:

\begin{itemize}
\item La interfaz no refleja correctamente la importancia de algunos elementos que no deberían pasar muy desapercibidos, como aquellos que sirven para indicar la popularidad de un libro.
\item No se ofrece mucha libertad a la hora de explorar libros. Sería interesante una sección dedicada a buscar libros según géneros, popularidad, puntuación, etc. configurando algún tipo de filtro más complejo.
\item Al consultar los libros relacionados con uno en concreto, no especifican por qué están relacionados con estos salvo por \textit{``A muchos usuarios también les gustó este otro libro``}. Más detalles del por qué están relacionados ayudarían a un usuario a tomar una decisión.
\item La abundancia de funcionalidades \textit{``Nice to have``} como los \textit{challenges}, \textit{quizzes}, etc. distraen al usuario de la funcionalidad principal.
\end{itemize}

\section{En \textit{My Many Reads}}

Considerando los puntos débiles y fuertes de estas plataformas (principalmente \textit{Goodreads} ya que es la más notable) se puede elaborar una lista de los elementos principales que resultarán de mayor utilidad para el inicio del ciclo de vida, ya que otros, como los elementos de gamificación, solo resultarían interesantes cuando se alcance cierta cantidad de usuarios activos en la plataforma, y por tanto no es necesario considerarlos de momento.

\begin{table}[H]
    \begin{tabularx}{\linewidth}{|X|X|}
        \hline
        \textbf{Apartado de interés} & \textbf{Propuesta de mejora} \\
        \hline
        Biblioteca gestionable     & La utilidad de mayor importancia en este tipo de aplicaciones. Tanto en \textit{Goodreads} como en \textit{Librarything} estas son similares, y difícilmente mejorables. Se propone mantener las características ofrecidas por estas con una interfaz mejorada y la opción adicional de incluir comentarios privados acerca de cada libro, que sirvan al lector como anotaciones para el futuro incluyendo aspectos negativos y positivos que sean de utilidad para decidir si volver a realizar la lectura. \\ 
        \hline
        Sistema de recomendaciones & Las recomendaciones a partir de la biblioteca solo se pueden mejorar con un ajuste del algoritmo. Dado que este es privado para cada plataforma, habrá que construir uno nuevo realizando un estudio de los parámetros a considerar y el peso de estos para el resultado. Adicionalmente se pretende ofrecer una nueva modalidad de recomendaciones en la que los usuarios relacionen dos libros justificando dicha relación, permitiendo a otros usuarios votar por esta y consultarlas por popularidad para obtener una información más relevante. \\ 
        \hline
        Motor de búsqueda   & En ambas plataformas el sistema de búsqueda de libros no permite una gran libertad a la hora de consultar libros en general en base a ciertos criterios de búsqueda, probablemente porque confíen en que solo se quiera buscar en base a las recomendaciones. Sin embargo, es posible que alguien quiera salir un poco de su zona de comfort de lectura y explorar nuevos géneros. Esto demuestra la utilidad de un sistema de búsqueda con capacidades más complejas de filtrado, combinando características como popularidad y géneros, autores o incluso fechas de publicación. \\ 
        \hline
    \end{tabularx}
\end{table}
	\chapter{Planificación}

En esta sección se recoge todo lo necesario para comprender cómo se planifica el desarrollo de este proyecto, describiendo las diferentes unidades de trabajo y sus características, enumerando las diferentes tareas planificadas y especificando la organización temporal del proyecto.

Así mismo, se explica la metodología seguida ya que de ésta depende, en parte, la planificación elaborada.

\section{Metodología utilizada}

Se pretende seguir una metodología de desarrollo ágil, elaborando tareas de pequeño tamaño y agrupándolas para cumplir en conjunto un objetivo de relativa importancia. Estas tareas se asignan a un \textit{sprint} de una duración concreta, permitiendo así tener claramente determinadas unidades de trabajo pequeñas y concretas, conformando funcionalidades u objetivos más complejos que se proponen cumplimentar en un intervalo de tiempo determinado.

Pero como en todo, pueden darse imprevistos durante el desarrollo que requieran de un tiempo adicional, por lo que esta planificación no es final en ningún momento, y siempre se debe contemplar la posibilidad de encontrarse con obstáculos no planificados que afecten al desarrollo correspondiente a un \textit{sprint}.

Para tener en cuenta apropiadamente estos imprevistos, como ya se ha mencionado anteriormente, no se crean y planifican tareas inmutables. A mayor avance en el tiempo más grande es la incertidumbre, por lo que se proponen una serie de objetivos poco detallados a cumplir en un largo periodo de tiempo, pero las tareas concretas para alcanzar estos objetivos solo se planifican en el momento en el que se vayan a realizar.

Como ya se comentará más adelante, el proyecto se está desarrollando en GitHub\cite{Github}, que ofrece herramientas que permiten elaborar una planificación, como \textit{issues} y \textit{milestones}, e incluso ofrece un tablero de KanBan y un \textit{backlog} de tareas en la sección de proyectos.

\subsection{Issues}

Las issues recogen el motivo u objetivo de un conjunto de cambios, tareas o acciones a realizar para contribuir al desarrollo o mantenimiento del proyecto. Por sí solas no ofrecen una gran versatilidad, pero gracias al sistema de etiquetas personalizables, a proyectos (para el tablero de KanBan y el backlog) y milestones empleadas en conjunto, se consigue un buen entorno de planificación.

La intención es que todos los cambios realizados en el proyecto (salvo quizás alguna excepción puntual) vayan asignados a un \textit{issue}, y que cada \textit{issue} esté asignado a un \textit{milestone} y/o haga referencia a otros issues en caso de considerar necesario subdividir en tareas aún más pequeñas.

Los tipos de \textit{issues} más destacables de mencionar son las historias de usuario (etiqueta \textit{user-story}). Éstas definen las historias de usuario, y por lo general están divididas en tareas, empleando otros issues que hagan referencia a la historia de usuario a la que van a contribuir.

\subsection{Milestones}

Cada \textit{milestone} representa un \textit{Producto Mínimamente Viable} o PMV. Un \textit{milestone} recoge todo el conjunto de tareas (\textit{issues}) necesarios de cumplimentar para alcanzar los requisitos propuestos del PMV propuesto. Del uso de estos \textit{milestones} como representaciones de PMVs se deduce fácilmente que se sigue una metodología que emplea iteraciones para diferentes entregas del proyecto en función de su desarrollo, es decir, cada nuevo \textit{milestone} se construye como un incremento del anterior, por lo que cada PMV debe suponer un incremento notable en la funcionalidad.

El hecho de que estos \textit{milestones} estén compuestos del conjunto de \textit{issues} necesarios para su terminación no quiere decir que absolutamente todos los \textit{issues} del proyecto estén asignados a un \textit{milestone}, puesto que como se ha explicado en la sección anterior, un \textit{issue} es una herramienta muy versátil, permitiendo instancias de éstos dedicadas a tareas cotidianas o de menor importancia que no encajen como parte de un \textit{milestone}, como por ejemplo la corrección de erratas.

Por lo general, un \textit{milestone} estará compuesto de historias de usuario o historias técnicas, pero eso no quiere decir que éstas (mayormente las historias de usuario) puedan formar parte de varios \textit{milestones}. Un ejemplo genérico de este caso sería el de la siguiente historia de usuario:

\begin{center}
    \textit{Como Usuario de My Many Reads,}\\
    \textit{quiero poder consultar recomendaciones personalizadas de libros,}\\
    \textit{para así ayudarme en la toma de decisión de mi próxima lectura.}
\end{center}

Esta historia se podría cumplimentar fácilmente con un algoritmo muy básico de recomendaciones en base a la lista de libros leídos por el usuario. Sin embargo, en el caso de \textit{My Many Reads}, esta historia necesita algo más que un algoritmo básico. Añadir complejidad al algoritmo para producir mejores resultados u ofrecer métodos alternativos de obtener recomendaciones también pueden ser parte de esta historia propuesta, pero no tendrían por qué serlo del mismo PMV, por lo que esta historia podría tener varios niveles de completitud, pudiendo pertenecer algunos de ellos a \textit{milestones} diferentes.

\section{Temporización}

Dado que el proyecto se está enfocando de manera ágil, no procede realizar una temporización completa de éste. Dicha temporización se elabora en varias ocasiones a lo largo del desarrollo, para periodos más cortos de tiempo, pero siempre con objetivos más difusos y abiertos al cambio en largos plazos de tiempo.

En primer lugar hay que categorizar los \textit{issues} de desarrollo del proyecto, para diferenciar entre varios tipos y asignar un tiempo a cada uno de éstos:

\begin{itemize}
    \item \textbf{Epopeyas:} representan un cambio notable en el proyecto. Son una agrupación de tareas, que se completa cuando la última de éstas se ha completado. Cada epopeya debe estar estimada a poder finalizarse en un periodo entre 5 y 8 días hábiles.
    \item \textbf{Tareas:} las unidades mínimas de trabajo. Por sí solas, pueden incluso no suponer ningún cambio en el proyecto. Se deben planificar para ser realizables en 1 a 3 días hábiles.
\end{itemize}

Con varias de estas epopeyas se definiría un \textit{milestone} que, como se indica en el apartado anterior, representa un PMV. Lo ideal sería que cada PMV se desarrollase en entre 4 y 6 semanas, lo que significa aproximadamente 4 epopeyas por PMV, pero es posible que en algunos casos se requerirá más o menos tiempo o epopeyas para completar un \textit{milestone}.

Cabe mencionar que también se emplearán issues que sean excepciones a estas categorías. Éstas pueden consistir en pequeños arreglos independientes, sin epopeya asignada, u objetivos de mayor duración que no encajan con la definición de epopeya pero que tampoco llegan a representar un PMV, quizás pudiendo definirse como \textit{fase}. Existirán pocas excepciones que no encajen en las categorías definidas previamente, y es por eso por lo que se prefiere simplemente dejarlas como excepciones y no intentar categorizar todas ellas.

\section{Seguimiento del desarrollo}

En la página del repositorio en GitHub se encuentran todas las \href{https://github.com/Anglepi/My-Many-Reads/issues}{\textit{issues}} y \href{https://github.com/Anglepi/My-Many-Reads/milestones}{\textit{milestones}} del proyecto, organizadas como ya se ha descrito anteriormente. Así mismo, en la sección de proyectos, se puede consultar un tablero \href{https://github.com/users/Anglepi/projects/1}{\textit{KanBan}} que contendrá tanto tareas como epopeyas de desarrollo y su estado actual, con el fin de apoyar la organización del desarrollo.

Dado que se están empleando los \href{https://github.com/Anglepi/My-Many-Reads/pulls}{\textit{Pull Requests}} para realizar los incrementos de progreso en el proyecto, éstos también son una fuente muy útil de información acerca del seguimiento y estado del desarrollo. Éstos son un buen lugar para encontrar la documentación del proyecto en un momento dado, ya que entre las tareas de integración continua \href{https://github.com/Anglepi/My-Many-Reads/actions/workflows/latex-build.yml}{hay una que genera este PDF}, tan solo selecciona el \textit{workflow} deseado y lo encontrarás abajo.

Adicionalmente, en la sección de \textit{Implementación} se enumerarán todos los PMV junto con sus epopeyas y las tareas que las forman, así como el tiempo dedicado a cada uno de estos elementos, que se irán actualizando conforme se desarrolle el proyecto.
	\chapter{Análisis del problema}

\section{\textit{User personas} y \textit{user journeys}}

Tanto los \textit{user personas}\cite{personas} como los \textit{user journeys}\cite{userjourney} son buenas técnicas para realizar un análisis en profundidad del problema y obtener más detalles acerca de cuales son los puntos críticos de la aplicación y cómo atenderlos de manera que se garantice una buena experiencia por parte del usuario. Estas herramientas no solo ayudan a comprender mejor el problema, sino también a elaborar una lista de requisitos funcionales y no funcionales, que son de gran ayuda a la hora de diseñar e implementar cualquier proyecto.

En primer lugar, una lista de \textit{personas} aportará una idea general de qué tipos de usuario podrían usar la plataforma solución planteada, además de, posiblemente, información adicional sobre problemas que pudieran existir y que radiquen en características concretas del usuario. A continuación se explorarán las historias de los usuarios a través de la plataforma, a través de \textit{user journeys}, con el fin de determinar qué soluciones desean encontrar y cómo pueden hacerlo, así como el nivel de satisfacción obtenido al realizar las acciones necesarias para conseguir su objetivo, influenciado por el proceso y el resultado. 

\subsection{\textit{Personas}}

\begin{table}[H]
    \centering
    \begin{tabularx}{\columnwidth}{|l|X|}
        \hline
        \textbf{Nombre y edad} & Sergio, 24 años. \\
        \hline
        \textbf{Rol} & Fan de la lectura (usuario lector). \\
        \hline
        \textbf{Ocupación} & Estudiante de Marketing Digital. \\
        \hline
        \textbf{Biografía} & Ya finalizando sus estudios, Sergio encuentra más tiempo que dedicar a la lectura. Disfruta tanto de leer novelas de fantasía y ciencia ficción como libros de historia. Sabe manejarse bien con el ordenador, aunque solo es experto en herramientas de comunicación y diseño, y no suele usar otro tipo de aplicaciones. Le gusta mantener una lista de objetivos a corto, medio y largo plazo ya que le ayuda a llevar una mejor calidad de vida. \\
        \hline
        \textbf{Temores} & Su mayor preocupación es que no aparezcan los libros de historia que le regaló su abuelo. Por otra parte, su principal preocupación es que la biblioteca no le permita diferenciar sus lecturas de historia de las novelas, ya que él lo considera dos tipos de \textit{hobby} diferentes. \\
        \hline
        \textbf{Necesidades} & Quiere una herramienta que gestione sus lecturas, permitiéndole diferenciar entre finalizadas, en progreso y pendientes con el fin de apoyar su lista de objetivos. Como es muy organizado, le gustaría que la herramienta fuese fácil de gestionar y versátil en caso de querer cambiar la organización de sus lecturas. \\
        \hline
    \end{tabularx}
\end{table}

\begin{table}[H]
    \centering
    \begin{tabularx}{\columnwidth}{|l|X|}
        \hline
        \textbf{Nombre y edad} & Ana, 32 años. \\
        \hline
        \textbf{Rol} & Fan de la lectura (usuario lector). \\
        \hline
        \textbf{Ocupación} & Talent recruiter en una compañía de IT. \\
        \hline
        \textbf{Biografía} & Desde su adolescencia Ana ha disfrutado muchísimo de la lectura, aficionándose a muchos géneros diferentes. De hecho en casa tiene una estantería llena de libros, la cual organiza de diferentes maneras continuamente, incluso por colores de la portada. Últimamente le resulta difícil encontrar un nuevo libro para leer, y se ha dado cuenta de que ha releído bastantes libros, por lo que quiere explorar nuevos libros, incluso los no disponibles en su idioma. \\
        \hline
        \textbf{Temores} & A Ana le preocupa que \textit{My Many Reads} no ofrezca buenas herramientas para descubrir nuevos libros. Piensa que la única manera de obtener recomendaciones de un sistema de esta naturaleza es a partir de una biblioteca, y que por tanto la plataforma no le ofrecerá opciones muy diferentes a otras herramientas del estilo. También quiere explorar libros con libertad en función de sus propios criterios, y le preocupa no tener mucha libertad a la hora de hacer esta búsqueda. \\
        \hline
        \textbf{Necesidades} & Su principal necesidad es encontrar nuevas lecturas. Para ello no quiere depender solo de recomendaciones en base a su biblioteca, si no que le gustaría disponer de herramientas para explorar manualmente libros en base a unos criterios de búsqueda específicos. También le gustaría encontrar recomendaciones en base a un libro concreto, y si están justificadas por otros usuarios, mejor. \\
        \hline
    \end{tabularx}
\end{table}

\begin{table}[H]
    \centering
    \begin{tabularx}{\columnwidth}{|l|X|}
        \hline
        \textbf{Nombre y edad} & Julio, 29 años. \\
        \hline
        \textbf{Rol} & Consumidor de información generada. \\
        \hline
        \textbf{Ocupación} & Product Manager del equipo de análisis de datos de una editorial. \\
        \hline
        \textbf{Biografía} & Julio comenzó a trabajar en su nuevo puesto hace poco. En este le han encargado encontrar información sobre el mercado actual del libro, para saber qué géneros y autores tienen más potencial de alcanzar un mayor nivel de ventas. Trabaja con muchos tipos de herramientas para obtener la información a analizar, y le gusta estar al día con las noticias de nuevas publicaciones. \\
        \hline
        \textbf{Temores} & La principal preocupación de Julio es hacer una inversión para obtener información de baja calidad, ya que esto aumenta su trabajo añadiendo requisitos de preprocesamiento de datos y, por si fuera poco, reduce la fiabilidad de estos, lo cual le puede llevar a tomar malas decisiones que afecten a su trabajo. Tampoco quiere datos procesados, ya que desconocerá las técnicas de procesamiento y tampoco serán fiables. \\
        \hline
        \textbf{Necesidades} & Quiere encontrar una buena fuente de datos sobre los libros más populares en la actualidad, para conocer los autores y géneros con más mercado y por tanto más posibilidades de buenas ventas. \\
        \hline
    \end{tabularx}
\end{table}

\subsection{\textit{User journeys}}

Para crear una biblioteca y añadir libros:

\begin{itemize}
    \item \textbf{Acciones}
    \begin{enumerate}
        \item Desde la página principal, acceder a la sección de \textit{Mis bibliotecas}.
        \item En la lista de bibliotecas, seleccionar la opción \textit{Crear nueva biblioteca}, indicando el nombre de esta.
        \item Con la biblioteca ya creada, acceder a la página de algún libro a través del buscador u otros medios.
        \item Desde la página del libro, buscar la opción \textit{Añadir a la biblioteca...} junto a los detalles del libro.
        \item En el menú emergente, seleccionar la nueva biblioteca.
    \end{enumerate}
\item \textbf{Puntos de contacto}
    \begin{itemize}
        \item Página principal.
        \item Lista de libros de un usuario.
        \item (Opcional) Buscador.
        \item Página de un libro.
    \end{itemize}
\item \textbf{Pensamientos}
    \begin{itemize}
        \item La creación de la biblioteca es un proceso sencillo.
        \item Los libros se añaden desde su página, y se accede a ellos de manera sencilla a través de un buscador en la barra de navegación presente en todas las vistas, lo cual resulta cómodo.
    \end{itemize}
\end{itemize}

Para buscar recomendaciones, existen varias formas:
\begin{itemize}
    \item \textbf{Acciones}
    \begin{enumerate}
        \item \textbf{Opción A.} Desde la página principal, acceder a la sección de \textit{Mis bibliotecas} y seleccionar la biblioteca cuyas recomendaciones se quieren consultar.
        \item En la página de la biblioteca, seleccionar la opción \textit{Recomendaciones}. Aparecerán recomendaciones generadas automáticamente a partir de los libros de la biblioteca.
        \item \textbf{Opción B.} Desde la página de un libro, acceder a la sección de \textit{Recomendaciones}. Aparecerán relaciones entre dicho libro y otros, realizadas y justificadas por otros usuarios.
    \end{enumerate}
\item \textbf{Puntos de contacto}
    \begin{itemize}
        \item Página principal.
        \item Lista de libros de un usuario.
        \item Lista de recomendaciones de una biblioteca.
        \item Página de un libro.
        \item Página de recomendaciones de otros usuarios en base a un libro.
    \end{itemize}
\item \textbf{Pensamientos}
    \begin{itemize}
        \item Las recomendaciones son fácilmente accesibles, y el algoritmo ofrece buenos resultados.
        \item Puedo consultar recomendaciones en base a mi libro favorito realizadas por otros usuarios con una justificación que me permite intuir si la recomendación es en base a mis aspectos favoritos del libro.
    \end{itemize}
\end{itemize}

\section{Aprovechamiento de la plataforma}
Si bien es cierto que ya existen algunas plataformas similares (ver \underline{\nameref{Estado del arte}}), se puede ver que no son perfectas, y de alguna manera se podrían introducir mejoras en algún sentido que podrían aumentar el número de usuarios, por lo que no sería descabellado pensar en hacerles competencia y tener algo de éxito.

El primer paso sería lograr una comunidad activa, para lo cual hay que ofrecer incentivos a los usuarios para emplear la plataforma, pues una red social sin usuarios no serviría para nada. Pero también hay que tener en cuenta que ofrecer un mero servicio a los usuarios sin obtener algún tipo de beneficio con el que cubrir los gastos de mantenimiento no lo más adecuado, por lo que habría que explotar este tráfico de usuarios e información de alguna manera:

\begin{itemize}
    \item Sirviendo como lugar publicitario para libros de todo tipo, ya que en la plataforma se concentrarían usuarios con interés en la lectura, y por tanto, posibles clientes.
    \item Así mismo, se podrían recoger datos sobre los gustos e intereses de los usuarios para que las propias editoriales sepan qué géneros y autores están más de moda, ayudándoles así a embarcarse en proyectos con mayor beneficio potencial.
\end{itemize} 

\section{Requisitos}

Con el apoyo de la información obtenida en las anteriores secciones se pueden definir unos requisitos que expresen las funcionalidades necesarias, los comportamientos esperados y los datos involucrados. Más adelante se elaborarán las diferentes historias de usuario para el desarrollo, enfocadas en cubrir todos los requisitos de esta lista.

A cada PMV planteado le corresponde un conjunto de requisitos, y dado que los PMV son incrementales, se espera que cada nuevo PMV cumpla los requisitos de todos los anteriores.

\subsection{\href{https://github.com/Anglepi/My-Many-Reads/milestone/2}{PMV-1}. Estructura de datos básica y base de la lógica de negocio}

\subsubsection{Requisitos de datos}
\begin{itemize}
    \item RD1-1 Almacenar libros con información básica acerca de estos.
    \item RD1-2 Almacenar bibliotecas, asociadas a sus dueños, formadas por una lista de entradas:
    \begin{itemize}
        \item RD1-2.1 Cada entrada estará formada por un libro, su estado y la puntuación ofrecida por el usuario.
    \end{itemize}
\end{itemize}

\subsubsection{Requisitos funcionales}
\begin{itemize}
    \item RF1-1 El sistema debe permitir la creación de bibliotecas.
    \item RF1-2 El sistema debe permitir entradas a una biblioteca, permitiendo libros duplicados.
    \item RF1-3 Se debe permitir consultar los libros y bibliotecas almacenadas.
\end{itemize}

En este primer PMV se desarrollarán las clases necesarias para representar las estructuras de datos que definen los libros, bibliotecas y sus entradas, al menos un modelo básico con el que poder ofrecer una funcionalidad mínima en este punto.

Las funciones básicas ofrecidas en este punto serán las de creación de una biblioteca, añadir entradas a esta y consultar información existente sobre libros y bibliotecas. Una peculiaridad en este punto es que se permite añadir múltiples veces el mismo libro a la misma biblioteca. Esto se debe a que controlar esta situación requiere de lógica adicional que no se ha planificado para este PMV principalmente por dos motivos: más adelante podrían surgir cambios en el modelo de datos o la funcionalidad deseada que alteren la complejidad y por tanto requieran de trabajo adicional para refactorizar esta lógica (en otras palabras, existe el riesgo de que esto se convierta en deuda técnica), y por otra parte se trata de un PMV, como su nombre indica debe incluir el mínimo de detalles que lo hagan viable.

En los futuros PMV se trabaja no solo en añadir nuevos componentes al sistema, sino además en extender la funcionalidad de los anteriormente existentes, por lo que no debe ser motivo de preocupación dejar ``cabos sueltos'' que atar en el futuro.

\subsection{\href{https://github.com/Anglepi/My-Many-Reads/milestone/3}{PMV-2}. Implementación de la API y sistema de recomendaciones}

\subsubsection{Requisitos de datos}
\begin{itemize}
    \item RD2-1 Almacenar relaciones entre dos libros.
    \begin{itemize}
        \item RD2-1.1 Almacenar comentarios de usuario y sus valoraciones asociados a una relación entre dos libros.
    \end{itemize}
\end{itemize}

\subsubsection{Requisitos funcionales}
\begin{itemize}
    \item RF2-1 El sistema debe permitir realizar operaciones de forma remota.
    \item RF2-2 Se debe permitir a los usuarios crear relaciones entre dos libros a modo de recomendación.
\end{itemize}

El objetivo de este segundo PMV es, en pocas palabras, introducir una API que publique las funcionalidades implementadas al exterior, de forma que estas sean accesibles de forma remota. El hecho de incluir un nuevo método de acceso a la funcionalidad podría no considerarse un incremento de esta, por lo que además se incluyen las recomendaciones entre dos libros generadas por usuarios, una pequeña adición que implica nuevos modelos de datos sencillos, y que aporta un componente de interés a la plataforma.

Incluir la API es un paso importante a nivel arquitectónico, ya que esta hará las veces de punto de contacto entre usuarios y/o sistemas externos con las funcionalidades de \textit{My Many Reads}, lo cual implica que cualquier funcionalidad incluida en la plataforma debe ser accesible directa o indirectamente desde la API.

\subsection{\href{https://github.com/Anglepi/My-Many-Reads/milestone/5}{PMV-3}. Base de datos remota}

\subsubsection{Requisitos funcionales}
\begin{itemize}
    \item RF3-1 El sistema debe ser capaz de realizar una conexión a una base de datos remota.
    \item RF3-2 Los cambios producidos por las transacciones de datos deben ser persistentes.
\end{itemize}

\subsubsection{Requisitos no funcionales}
\begin{itemize}
    \item RNF3-1 Los datos almacenados deben cumplir la legislación siguiendo las prácticas establecidas en esta.
\end{itemize}

Hasta este punto, \textit{My Many Reads} no empleaba una base de datos real, sino que almacenaba la información única y exclusivamente en entorno de ejecución. Esto, obviamente, representa una serie de problemas importantes, como una limitación grande en cuanto a cantidad de información o la imposibilidad de mantener dicha información entre diferentes ejecuciones (es decir, ausencia de persistencia).

Como los requisitos indican, en este PMV se van a incluir los cambios necesarios para que el sistema haga uso de una base de datos real. Esto implica una investigación sobre qué tipo de base de datos emplear, su compatibilidad con las tecnologías incluidas hasta este punto e introducir la lógica necesaria para sacarle partido.

A partir de este momento se gestionará la información de manera apropiada, en una base de datos que aporta persistencia y con una lógica que permite consistencia y transacciones coherentes de información. A nivel funcional esto sería lo mínimo necesario, sin embargo, quedan otros detalles que cubrir. Los datos almacenados deben cumplir las legislaciones pertinentes, en función de qué tipo de datos se almacenen. Esto por lo general suele implicar que la información introducida por un usuario deba ser eliminada si este así lo desea.

Esto supone un problema, puesto que toda esta información resulta de especial utilidad para la plataforma. Por una parte, dado su carácter social, tendría un impacto negativo la sensación de ausencia o decrecimiento de actividad de otros usuarios, pero por otro lado (y puede que más importante), el sistema de recomendaciones se vería afectado negativamente dado que depende, en parte, de la información generada por usuarios para generar mejores recomendaciones.

Para solventar esta problemática, se propone que la tarea de eliminar esta información suponga, en casos pertinentes, desvincularla totalmente de los usuarios creadores con el fin de que dicha información siga aportando utilidad a la plataforma. Obviamente, en el caso de datos de carácter personal si los hubiere, esta información será eliminada totalmente.

\subsection{\href{https://github.com/Anglepi/My-Many-Reads/milestone/4}{PMV-4}. Recolectar y ofrecer estadísticas}

\subsubsection{Requisitos funcionales}
\begin{itemize}
    \item RF4-1 El sistema debe permitir la consulta de visitas y nota media de libros.
    \item RF4-2 El sistema debe permitir la consulta de popularidad de géneros.
    \item RF4-3 El sistema debe permitir consultar un \textit{top} de libros según sus visitas o su nota media.
\end{itemize}

\subsubsection{Requisitos de datos}
\begin{itemize}
    \item RD4-1 Se debe almacenar el número de visitas realizadas a cada libro.
\end{itemize}

El proyecto debe tener algún elemento que le permita mantenerse. En este caso, se trata de información y estadísticas de uso, que se puede emplear en fines comerciales al visibilizar el estado del mercado actual indicando las tendencias más populares.

En este punto el sistema ya cuenta con las puntuaciones que los usuarios establecen en los libros a través de sus bibliotecas, por lo tanto el siguiente paso es agrupar todas estas puntuaciones para obtener una media referente a cada libro. Esto es algo que mediante consultas se puede conseguir, por lo que no se requieren cambios en la estructura de la base de datos.

Por otra parte se encuentra la popularidad, que consiste en el número de visitas que recibe cada libro. Esta información no se encuentra en la base de datos, por lo que habrá que editar la estructura de esta para incluirla. Adicionalmente, habrá que incluir alguna funcionalidad que incremente el número de visitas de un libro con cada consulta. Esto se puede lograr, dependiendo la base de datos, con operaciones o triggers definidos en esta, lo cual facilita la lógica de uso de este proceso, o mediante la redefinición de la consulta empleada para solicitar los datos de un libro, logrando que en este momento también se incrementen las visitas. Dada la volatilidad experimentada en el sistema externo de base de datos, que se comentará en la sección de implementación. La segunda opción parece más sensata.

Una vez obtenido cierto control sobre la información deseada, hay que ofrecer la posibilidad de consultarla en formatos que puedan resultar interesantes. Una selección de formatos interesantes incluye, además de la más básica consulta de un libro mostrando su puntuación media y visitas, un \textit{top} de géneros más populares y un \textit{top} de libros según visitas o popularidad.

Este PMV está establecido principalmente como un primer paso hacia la satisfacción de los objetivos propuestos para los \href{https://github.com/Anglepi/My-Many-Reads/blob/main/docs/md/personas/information-consumer.md}{usuarios consumidores de información generada}.

\section{Historias de usuario}

Las historias de usuario se usan para definir los beneficios que ciertos tipos de usuarios desean obtener en determinadas situaciones. Cada una de estas historias de usuario representa un avance en la funcionalidad del sistema, y su desarrollo se compone de una a varias tareas.

Partiendo de los requisitos previamente listados, refinándolos en historias de usuario, se puede dar paso a la elaboración de tareas que guíen la implementación del proyecto. Cabe aclarar que, como este desarrollo sigue una mentalidad ágil, las historias y tareas se han ido generando a corto plazo, por lo que la siguiente lista de historias de usuario y sus tareas se ha ido construyendo a lo largo del proceso de desarrollo.

Algunas historias, como ya se verá a continuación, son técnicas. Esto se debe a que lo que se pretende lograr con ellas no encaja del todo en la definición de historia de usuario, pero sin embargo se dividen en tareas igualmente y su implementación es lo suficientemente relevante como para mencionarlas como parte del desarrollo de un PMV.

Así mismo cabe mencionar que el primer \textit{milestone} del proyecto se excluye de esta sección, pues está compuesto únicamente de documentación no directamente relacionada con las tareas de implementación.

\subsection{PMV-1. Gestión básica de libros y bibliotecas.}

\begin{itemize}
    \item \href{https://github.com/Anglepi/My-Many-Reads/issues/29}{\textbf{HU-1}} Solicitar información de libros \\
    Como Sergio, lector en MMR, \\
    quiero solicitar información sobre libros de una base de datos, \\
    con el fin de saber si me pueden interesar según sus características.
    \begin{itemize}
        \item \href{https://github.com/Anglepi/My-Many-Reads/issues/31}{\textbf{T-1.1}} Definir estructura básica de datos para los libros.
        \item \href{https://github.com/Anglepi/My-Many-Reads/issues/32}{\textbf{T-1.2}} Crear primera aproximación de BD de libros.
        \item \href{https://github.com/Anglepi/My-Many-Reads/issues/33}{\textbf{T-1.3}} Métodos para tratar las propiedades de los libros.
    \end{itemize}
    \item \href{https://github.com/Anglepi/My-Many-Reads/issues/30}{\textbf{HU-2}} Bibliotecas para gestionar lecturas
    Como Sergio, lector en MMR, \\
    quiero diferentes colecciones de libros de mi elección, \\
    con el fin de gestionar mis lecturas bajo mis propios criterios.
    \begin{itemize}
        \item \href{https://github.com/Anglepi/My-Many-Reads/issues/34}{\textbf{T-2.1}} Definir estructura básica de datos para las bibliotecas.
        \item \href{https://github.com/Anglepi/My-Many-Reads/issues/35}{\textbf{T-2.2}} Funcionalidad básica de gestión de bibliotecas.
    \end{itemize}
\end{itemize}

\subsection{PMV-2. Implementación de la API y sistema de recomendaciones}
\label{pmv2}

\begin{itemize}
    \item \href{https://github.com/Anglepi/My-Many-Reads/issues/45}{\textbf{HT-3}} Creación de la API. \\
    Es necesario tener una API para realizar todas las tareas de consulta y tratamiento de datos pertinentes sobre las entidades del sistema. \\
    Se utilizara tanto para alimentar el sistema web que se desarrolle en el futuro como para \\
    permitir a los consumidores de información realizar sus consultas.
    \begin{itemize}
        \item \href{https://github.com/Anglepi/My-Many-Reads/issues/46}{\textbf{T-3.1}} Escoger un framework para la API e implementar un ejemplo de uso.
        \item \href{https://github.com/Anglepi/My-Many-Reads/issues/47}{\textbf{T-3.2}} \textit{Endpoints} para los libros.
        \item \href{https://github.com/Anglepi/My-Many-Reads/issues/48}{\textbf{T-3.2}} \textit{Endpoints} para las bibliotecas.
    \end{itemize}
    \item \href{https://github.com/Anglepi/My-Many-Reads/issues/49}{\textbf{HU-4}} Sistema de recomendaciones manuales. \\
    Como Sergio, lector en MMR, \\
    quiero relacionar dos libros mediante una recomendación personal, \\
    con el fin de ayudar a otros usuarios a elegir sus lecturas.
    \begin{itemize}
        \item \href{https://github.com/Anglepi/My-Many-Reads/issues/57}{\textbf{T-4.1}} Determinar e implementar la mejor estructura de datos para recomendaciones creadas por usuarios.
        \item \href{https://github.com/Anglepi/My-Many-Reads/issues/58}{\textbf{T-4.2}} Implementar funcionalidades requeridas para las recomendaciones creadas por usuarios.
        \item \href{https://github.com/Anglepi/My-Many-Reads/issues/59}{\textbf{T-4.3}} Habilitar funcionalidades de las recomendaciones creadas por usuarios en la API.
    \end{itemize}
    \item \href{https://github.com/Anglepi/My-Many-Reads/issues/50}{\textbf{HU-5}} Sistema de recomendaciones automático. \\
    Como Sergio, lector en MMR, \\
    quiero consultar recomendaciones en base a una de mis bibliotecas, \\
    con el fin de ayudarme a decidir mis lecturas.
    \begin{itemize}
        \item \href{https://github.com/Anglepi/My-Many-Reads/issues/66}{\textbf{T-5.1}} Encontrar una estructura de datos que represente la recopilación de características de los libros de una biblioteca.
        \item \href{https://github.com/Anglepi/My-Many-Reads/issues/67}{\textbf{T-5.2}} Medir el nivel de correlación entre un libro y una biblioteca, de acuerdo a sus características recopiladas, para calcular su nivel de similitud.
        \item \href{https://github.com/Anglepi/My-Many-Reads/issues/68}{\textbf{T-5.3}} Crear una lista de recomendaciones a partir de una librería y un conjunto de libros.
    \end{itemize}
\end{itemize}

\subsection{PMV-3. Base de datos remota}
\label{pmv3}

\begin{itemize}
    \item \href{https://github.com/Anglepi/My-Many-Reads/issues/76}{\textbf{HT-6}} Capa de gestión de datos. \\
    Es necesario incluir un nuevo componente que haga las veces de conector hacia una base de datos remota. \\
    Este componente debería encargarse de gestionar todas las consultas a la base de datos, ya que no es responsabilidad de la API.
    \begin{itemize}
        \item \href{https://github.com/Anglepi/My-Many-Reads/issues/77}{\textbf{T-6.1}} Crear la estructura básica del gestor de datos.
        \item \href{https://github.com/Anglepi/My-Many-Reads/issues/78}{\textbf{T-6.2}} Añadir al gestor la funcionalidad relevante para las consultas sobre libros.
        \item \href{https://github.com/Anglepi/My-Many-Reads/issues/79}{\textbf{T-6.3}} Añadir al gestor la funcionalidad relevante para las consultas sobre bibliotecas.
        \item \href{https://github.com/Anglepi/My-Many-Reads/issues/80}{\textbf{T-6.4}} Añadir al gestor la funcionalidad relevante para las consultas sobre recomendaciones.
    \end{itemize}
    \item \href{https://github.com/Anglepi/My-Many-Reads/issues/76}{\textbf{HT-7}} Base de datos persistente. \\
    Hasta ahora la base de datos se inicializaba de nuevo con cada ejecución. \\
    Para lograr persistencia, se propone implementar el uso de una base de datos externa que almacene y recupere la información en memoria secundaria.
    \begin{itemize}
        \item \href{https://github.com/Anglepi/My-Many-Reads/issues/77}{\textbf{T-7.1}} Encontrar posibles opciones y escoger la mejor de ellas.
        \item \href{https://github.com/Anglepi/My-Many-Reads/issues/78}{\textbf{T-7.2}} Crear un script que defina las estructuras de las tablas en la base de datos, así como algunos datos de prueba.
        \item \href{https://github.com/Anglepi/My-Many-Reads/issues/79}{\textbf{T-7.3}} Modificar el gestor de datos para que realice la conexión a esta nueva base de datos.
    \end{itemize}
    \item \href{https://github.com/Anglepi/My-Many-Reads/issues/98}{\textbf{HU-8}} Evitar posibles inconsistencias. \\
    Como Sergio, lector en MMR, \\
    quiero que la información y funcionalidades sean consistentes tras cada operación, \\
    con el fin de evitar situaciones inesperadas causadas por mis acciones.
    \begin{itemize}
        \item \href{https://github.com/Anglepi/My-Many-Reads/issues/93}{\textbf{T-8.1}} Evitar entradas duplicadas en las bibliotecas.
        \item \href{https://github.com/Anglepi/My-Many-Reads/issues/94}{\textbf{T-8.2}} Añadir información de cara al usuario sobre el tratamiento de datos en \textit{My Many Reads}.
        \item \href{https://github.com/Anglepi/My-Many-Reads/issues/96}{\textbf{T-8.3}} Usar un identificador único para identificar los libros, evitando el uso del ISBN.
    \end{itemize}
\end{itemize}

\subsection{PMV-4. Recolectar y ofrecer estadísticas}

\begin{itemize}
    \item \href{https://github.com/Anglepi/My-Many-Reads/issues/116}{\textbf{HU-9}} Estadísticas de popularidad de los libros.  \\
    Como Julio, PM de una editorial, \\
    quiero consultar las tendencias populares actuales, \\
    con el fin de tener una visión del mercado actual.
    \begin{itemize}
        \item \href{https://github.com/Anglepi/My-Many-Reads/issues/117}{\textbf{T-9.1}} La base de datos debe permitir almacenar estadísticas.
        \item \href{https://github.com/Anglepi/My-Many-Reads/issues/118}{\textbf{T-9.2}} El gestor de datos debe permitir registrar y consultar las estadísticas de libros y géneros populares.
        \item \href{https://github.com/Anglepi/My-Many-Reads/issues/119}{\textbf{T-9.3}} La API necesita nuevos endpoints para consultar estadísticas.
    \end{itemize}
\end{itemize}
	\chapter{Implementación}

En este capítulo se describe el proceso de implementación junto con todos los elementos en los que éste se apoya, una descripción de todas las elecciones técnicas tomadas junto con su justificación.

En primer lugar se muestra el listado de las historias de usuario generadas junto con sus correspondientes tareas, que dan forma a las unidades de trabajo refinadas. Tras esto, se encuentran todos los dilemas tecnológicos considerados, junto con la solución seleccionada y  una lista de alternativas brevemente definidas con el fin de proveer una justificación fundada y válida de la elección tomada.

\subsection{Historias de usuario}

Partiendo de los requisitos previamente listados, refinándolos en historias de usuario, se puede dar paso a la elaboración de tareas que guíen la implementación del proyecto. Cabe aclarar que, como este desarrollo sigue una mentalidad ágil, las historias y tareas se han ido generando a corto plazo, por lo que la siguiente lista de historias de usuario y sus tareas se ha ido construyendo a lo largo del proceso de desarrollo.

Algunas historias, como ya se verá a continuación, aparecen repetidas. Esto se debe a que su desarrollo ha abarcado más de un PMV, pero en cada uno de estos se tratan diferentes tareas, por lo que solo se citaran las que correspondan al PMV descrito.

Así mismo cabe mencionar que el primer \textit{milestone} del proyecto se excluye de esta sección, pues está compuesto únicamente de documentación no directamente relacionada con las tareas de implementación.

\subsubsection{PMV-1. Estructura de datos básica y base de la lógica de negocio}

\begin{itemize}
    \item \href{https://github.com/Anglepi/My-Many-Reads/issues/7}{\textbf{HU-1}} Criterios de evaluación del jurado \\
    Como miembro del jurado, \\
    quiero recibir una documentación y presentación sobre este proyecto y su desarrollo, \\
    con el fin de poder evaluarlo en base a unos criterios específicos.

    \item \href{https://github.com/Anglepi/My-Many-Reads/issues/29}{\textbf{HU-2}} Solicitar información de libros \\
    Como Sergio, lector en MMR, \\
    quiero solicitar información sobre libros de una base de datos, \\
    con el fin de saber si me pueden interesar según sus características.
    \begin{itemize}
        \item \href{https://github.com/Anglepi/My-Many-Reads/issues/31}{\textbf{T-2.1}} Definir estructura básica de datos para los libros.
        \item \href{https://github.com/Anglepi/My-Many-Reads/issues/32}{\textbf{T-2.2}} Crear primera aproximación de BD de libros.
        \item \href{https://github.com/Anglepi/My-Many-Reads/issues/33}{\textbf{T-2.3}} Métodos para tratar las propiedades de los libros.
    \end{itemize}

    \item \href{https://github.com/Anglepi/My-Many-Reads/issues/30}{\textbf{HU-3}} Bibliotecas para gestionar lecturas
    Como Sergio, lector en MMR, \\
    quiero una biblioteca de libros de mi elección, \\
    con el fin de gestionar mis lecturas.
    \begin{itemize}
        \item \href{https://github.com/Anglepi/My-Many-Reads/issues/34}{\textbf{T-3.1}} Definir estructura básica de datos para las bibliotecas.
        \item \href{https://github.com/Anglepi/My-Many-Reads/issues/35}{\textbf{T-3.2}} Funcionalidad básica de gestión de bibliotecas.
    \end{itemize}
\end{itemize}
	\chapter{Costes del proyecto}

Anteriormente en la sección de planificación se mostró una descripción de la \hyperref[Temporizacion]{\underline{temporización del proyecto}}. Dado que se sigue una filosofía ágil para el proyecto, al principio de este era inviable plantear una planificación temporal completa.

Es por esto que una vez realizado todo el desarrollo, se puede elaborar una estimación de costes más precisa, tomando como referencia la estimación media de tiempo por unidad de trabajo, una aproximación de salario de un informático y la cantidad de tareas realizadas.

En total, se han desarrollado \textbf{10 epopeyas}, las cuales se intentaron definir para completar su desarrollo en 7 días, pero que finalmente se extendieron algunas de ellas hasta los 9 días, debido principalmente a la falta de experiencia con ciertas tecnologías y algunos problemas imprevistos. Adicionalmente, para cada epopeya, hubo que dedicar una media de 3 días en revisiones

Adicionalmente, surgieron \textbf{5 \textit{bugs}}, tres de los cuales eran de relativa simplicidad y se resolvieron en un día cada uno, mientras que los otros dos implicaron algo más de trabajo, dos días por cada uno.

Existen otras tareas, agrupadas como \textit{Architecture and Quality}, independientes y definidas para mejorar el proceso de desarrollo, como por ejemplo refactorizaciones, y otras para trabajar en integración continua, \textit{CI}. En total se ha trabajado en \textbf{5 tareas de \textit{Architecture and Quality}} y \textbf{3 tareas de \textit{CI}}, con una media aproximada de un día por tarea.

Tomando finalmente una aproximación de 100 días para las epopeyas y otros 12 días para tareas inependientes de desarrollo, con una media de 4 horas por día de trabajo, se llegan a las 452 horas dedicadas al desarrollo, mantenimiento y documentación del código.

Existen también otras tareas no contabilizadas en el total anterior, como la creación, configuración y gestión del repositorio, del entorno local de desarrollo, y otras tareas de documentación e investigación como el análisis del problema, el estado del arte y la definición de planificación a emplear. Teniendo esto en cuenta, el tiempo empleado al desarrollo completo del proyecto hasta este punto puede llegar a una aproximación de 500 horas.

Dado que soy un único desarrollador, y que estimo que un salario promedio en el lugar donde vivo para un informático con experiencia ronda los 15€ netos a la hora, lo cual asciende a un total de 7500€ netos únicamente como pago hacia un desarrollador.

Por suerte, todas las tecnologías empleadas hasta el momento han resultado ser de uso libre, por lo que no fue necesaria una inversión económica adicional en este aspecto. Sin embargo, también es necesario contabilizar gatos en material de oficina, equipamiento y otros.

\begin{table}[H]
    \begin{tabular}{ll}
        \multicolumn{2}{c}{\textbf{GASTOS NETOS}}               \\
        \rowcolor[HTML]{C0C0C0} 
        \textbf{Gastos de personal}           & \textbf{7500,00 euros} \\
        Personal contratado                   & 7500,00 euros          \\
        Subcontrataciones / colaboraciones    & 0,00 euros             \\
        \rowcolor[HTML]{C0C0C0} 
        \textbf{Gastos directos de ejecución} & \textbf{1466,50 euros} \\
        Equipamiento tecnológico              & 1450,00 euros          \\
        Material de oficina                   & 5,00 euros             \\
        Viajes y dietas                       & 11,50 euros            \\ \hline
        \rowcolor[HTML]{C0C0C0} 
        \textbf{Total}                        & \textbf{8966,50 euros}
    \end{tabular}
\end{table}
	\chapter{Conclusiones, estado actual y futuro del proyecto}

En este último capítulo se recogerán las conclusiones obtenidas tras el desarrollo del proyecto hasta este punto, se resumirá el estado actual, con todas las características implementadas y se comentarán los pasos a seguir en etapas futuras de desarrollo.

\section{Conclusiones}

No existe una razón concreta por la que yo quise realizar un proyecto relacionado con libros, sino más bien mi intención fue la de poner a prueba mi capacidad para organizar un proyecto, así como emplear tecnologías que en mi trabajo no uso, con el fin de aprender un poco más y salir de mi zona de comfort.

Gracias a esto he aprendido bastante, y no solo con lo finalmente implementado en el proyecto sino con las numerosas alternativas que se consideraron y se pusieron a prueba en pequeños PoC (\textit{Proof of Concept}). Esto me ha permitido hacer una pequeña visita al estado de algunas tecnologías que se usan actualmente y me ha permitido contrastarlo con mi conocimiento previo, por lo que ahora me veo con una mejor capacidad de toma de decisiones al haber explorado más a fondo nuevas opciones.

También me ha permitido reafirmar algunas sospechas sobre ciertas formas de organizar el trabajo. Comencé pensando, erróneamente, que al ser un único desarrollador, el tamaño de las epopeyas y tareas refinadas debía más pequeño. Tras un primer intento de aplicar esta filosofía me di cuenta de que los costes en tiempo de la gestión administrativa de estas tareas empezaban a ser bastante altos en proporción al tiempo necesario para resolver dichas tareas, es decir, programar.

Por otra parte, los beneficios de esta gestión administrativa tienen menos impacto al tratarse de un trabajo en solitario. Al ser un único desarrollador, lo más común es tener el conocimiento y los objetivos de cada historia bastante claros, y parte de la intención de refinar y explicar apropiadamente cada una de las tareas a desarrollar es que otros desarrolladores puedan fácilmente trabajar en ellas sin tener que buscar información adicional del contexto de la tarea.

También me ha sorprendido la utilidad que tiene, a nivel de revisión, abrir \textit{pull requests} cuando solo hay un desarrollador. Aunque es cierto que mi tutor los revisaba, también me daba tiempo a ``aparcar'' la tarea un tiempo y desconectar, permitiéndome así revisar mi propio trabajo.

En cuanto a tecnologías, he aprendido bastante al estudiar y poner en práctica muchas de estas a las cuales no estoy acostumbrado. Un ejemplo de esto son los \textit{ORM}, de los cuales no conocía su existencia, y me parece una opción muy buena para determinados tipos de aplicaciones. Por otra parte, \textit{python} ha sido desde siempre un lenguaje que nunca me ha terminado de gustar como ya expliqué al momento de tomar esta decisión, pero al ver la cantidad de complementación que un \textit{linter} puede aportarle, como por ejemplo las anotaciones de tipo, ha hecho que mi opinión mejore, aunque sigue teniendo características que no me terminan de gustar como la complejidad de la gestión de los entornos de desarrollo.

En resumen, este proyecto me ha permitido aprender sobre diferentes tipos de bases de datos, la implementación de sistemas de integración continua, de los cuales solo me había servido de su utilidad anteriormente, y de la importancia de la configuración del entorno de desarrollo para acomodar el trabajo. Por otra parte, he podido experimentar el cierre del servicio de la base de datos que estaba usando, lo cual me ha permitido experimentar de primera mano que la migración es un proceso realmente simple, y que aunque estas situaciones resulten molestas, lidiar con ellas acaba siendo bastante sencillo.

\section{Estado actual}

\section{El futuro del proyecto}
	
	\newpage
	\bibliography{biblio}
	\bibliographystyle{plain}
	
\end{document}

