\chapter{Implementación}

En este capítulo se describe el proceso de implementación junto con todos los elementos en los que éste se apoya, una descripción de todas las elecciones técnicas tomadas junto con su justificación.

En primer lugar se muestra el listado de las historias de usuario generadas junto con sus correspondientes tareas, que dan forma a las unidades de trabajo refinadas. Tras esto, se encuentran todos los dilemas tecnológicos considerados, junto con la solución seleccionada y  una lista de alternativas brevemente definidas con el fin de proveer una justificación fundada y válida de la elección tomada.

\subsection{Historias de usuario}

Partiendo de los requisitos previamente listados, refinándolos en historias de usuario, se puede dar paso a la elaboración de tareas que guíen la implementación del proyecto. Cabe aclarar que, como este desarrollo sigue una mentalidad ágil, las historias y tareas se han ido generando a corto plazo, por lo que la siguiente lista de historias de usuario y sus tareas se ha ido construyendo a lo largo del proceso de desarrollo.

Algunas historias, como ya se verá a continuación, aparecen repetidas. Esto se debe a que su desarrollo ha abarcado más de un PMV, pero en cada uno de estos se tratan diferentes tareas, por lo que solo se citaran las que correspondan al PMV descrito.

Así mismo cabe mencionar que el primer \textit{milestone} del proyecto se excluye de esta sección, pues está compuesto únicamente de documentación no directamente relacionada con las tareas de implementación.

\subsubsection{PMV-1. Estructura de datos básica y base de la lógica de negocio}

\begin{itemize}
    \item \href{https://github.com/Anglepi/My-Many-Reads/issues/7}{\textbf{HU-1}} Criterios de evaluación del jurado \\
    Como miembro del jurado, \\
    quiero recibir una documentación y presentación sobre este proyecto y su desarrollo, \\
    con el fin de poder evaluarlo en base a unos criterios específicos.
    \begin{itemize}
        \item \href{https://github.com/Anglepi/My-Many-Reads/issues/15}{\textbf{T-1.1}} Definir estructura básica de datos para los libros.
        \item \href{https://github.com/Anglepi/My-Many-Reads/issues/36}{\textbf{T-1.2}} Crear primera aproximación de BD de libros.
        \item \href{https://github.com/Anglepi/My-Many-Reads/issues/37}{\textbf{T-1.3}} Añadir requisitos para el segundo PMV.
    \end{itemize}
    \item \href{https://github.com/Anglepi/My-Many-Reads/issues/29}{\textbf{HU-2}} Solicitar información de libros \\
    Como Sergio, lector en MMR, \\
    quiero solicitar información sobre libros de una base de datos, \\
    con el fin de saber si me pueden interesar según sus características.
    \begin{itemize}
        \item \href{https://github.com/Anglepi/My-Many-Reads/issues/31}{\textbf{T-2.1}} Definir estructura básica de datos para los libros.
        \item \href{https://github.com/Anglepi/My-Many-Reads/issues/32}{\textbf{T-2.2}} Crear primera aproximación de BD de libros.
        \item \href{https://github.com/Anglepi/My-Many-Reads/issues/33}{\textbf{T-2.3}} Métodos para tratar las propiedades de los libros.
    \end{itemize}
    \item \href{https://github.com/Anglepi/My-Many-Reads/issues/30}{\textbf{HU-3}} Bibliotecas para gestionar lecturas
    Como Sergio, lector en MMR, \\
    quiero una biblioteca de libros de mi elección, \\
    con el fin de gestionar mis lecturas.
    \begin{itemize}
        \item \href{https://github.com/Anglepi/My-Many-Reads/issues/34}{\textbf{T-3.1}} Definir estructura básica de datos para las bibliotecas.
        \item \href{https://github.com/Anglepi/My-Many-Reads/issues/35}{\textbf{T-3.2}} Funcionalidad básica de gestión de bibliotecas.
    \end{itemize}
\end{itemize}

\subsection{Entorno de desarrollo}

A continuación se describen todos los elementos que conforman el entorno de desarrollo, justificando las elecciones tomadas enfrentándolas a las alternativas encontradas e indicando el por qué de la decisión en base a la naturaleza de este proyecto.

\subsubsection{Gestor de tareas}

\textbf{Elección final:} \href{https://www.gnu.org/software/make/}{\textit{make}}.

\textit{Make} es uno de los gestores de tareas más populares actualmente, muy probablemente debido a la sencillez de su uso e instalación y su versatilidad. El factor determinante para tomar esta decisión fue, principalmente, esta versatilidad, pues sabía que no solo lo usaría para gestionar actividades del desarrollo, sino para todo lo relacionado con la elaboración de este proyecto, incluida la documentación.

Esta elección fue de las primeras que hubo de tomarse, mucho antes de comenzar el desarrollo, pues como es lo habitual, el proyecto comenzó por una documentación e investigación que sirviera para justificar la viabilidad del mismo, sin ser muy relevantes los detalles técnicos del mismo.

Solo es necesario un fichero \href{https://github.com/Anglepi/My-Many-Reads/blob/main/Makefile}{\textit{Makefile}} en el que incluir las tareas a automatizar bajo sus nombres

Las alternativas a considerar se investigaron una vez se tomó la elección del lenguaje de programación a usar, ya que el resto de gestores de tareas genéricos con los que me he topado los descartaba, principalmente debido a la sintaxis que empleaban o a que requerían la instalación de complementos adicionales que, probablemente, no usaría para otra cosa, por lo que me pareció inconveniente.

\textbf{Alternativa principal:} \href{https://www.pyinvoke.org/}{\textit{invoke}}.

Es una muy buena herramienta para gestión de tareas en Python. Mediante un fichero \textit{tasks.py} se definen las tareas a realizar, y dado que la sintaxis de python es muy conocida y para nada compleja, no es difícil de construir. Además se pueden ejecutar órdenes en la línea de comandos muy fácilmente, por lo que también es muy versátil.

Muy probablemente habría sido la opción principal si se supiera desde el principio que se usaría Python para desarrollar el proyecto, dado que no tiene absolutamente nada que envidiarle a \textit{make}. Sin embargo éste se usaba desde el principio del proyecto, y dado que el cambio no aportaba ningún beneficio, se optó por descartar la opción.

\subsubsection{Editor de código}

\textbf{Elección final:} \href{https://code.visualstudio.com/}{\textit{Visual Studio Code}}.

Es el editor más potente y, desgraciadamente, eso se nota cuando se utiliza en máquinas de bajas prestaciones. Sin embargo, ofrece muchísimas opciones de configuración base, detecta una gran cantidad de lenguajes a los que provee de un formato visual de apoyo y es propiedad de \textit{Microsoft}, por lo que se espera un buen mantenimiento y actualizaciones frecuentes.

Destaca principalmente por ofrecer de forma nativa una terminal, que aunque por defecto usa la nativa del sistema en el que se encuentre instalado, es muy sencillo instalar y usar cualquier otra. Además existe una gama muy amplia de complementos gratuitos y desarrollados por la comunidad, muy fáciles de instalar y gestionar. 

El propio editor te ofrece sugerencias cuando detecta el entorno en el que estás trabajando, y ni si quiera es necesario recurrir al navegador web para consultar información de éstos o instalarlos, pues su gestor de complementos ya satisface todas estas necesidades. Algunos de estos complementos útiles para el proyecto son:
\begin{itemize}
    \item \href{https://marketplace.visualstudio.com/items?itemName=ms-python.python}{Python}, que ofrece características propias del lenguaje como \textit{linting}, \textit{debugging} y formateo de código.
    \item \href{https://marketplace.visualstudio.com/items?itemName=eamodio.gitlens}{GitLens}, que aumenta las funcionalidades relacionadas con Git en el editor.
\end{itemize}

\textbf{Alternativa principal:} \href{https://www.sublimetext.com/}{\textit{Sublime Text}}.

Es otro editor muy popular, y aunque no tenga un equipo tan grande como el de \textit{Visual Studio Code} detrás, recibe actualizaciones de manera frecuente con notables mejoras.

Una ventaja que tiene frente al anterior es que no consume tantos recursos, pero dado que tengo la suerte de trabajar en una máquina de buenas prestaciones, este aspecto realmente no me supone un beneficio.

Por otra parte, lo que me hizo descartar esta opción fue el proceso de búsqueda e instalación de complementos, que es más manual y tediosa, así como la ausencia de una terminal en el entorno de forma nativa.

Cabe destacar que existen otras alternativas, IDEs para Python, que también pueden considerarse, como \href{https://docs.python.org/3/library/idle.html}{\textit{IDLE}}, que viene integrado, o \href{https://www.jetbrains.com/es-es/pycharm/}{\textit{PyCharm}} de JetBrains, un equipo muy conocido por desarrollar muy buenos IDEs.

Un IDE sería ideal si solo se fuera a trabajar en código, pero dado que este proyecto también requiere de documentación y podría incluir una variedad de tecnologías a emplear, un simple editor de código con complementos es mucho más adecuado.

\subsubsection{Lenguaje de programación}

\textbf{Elección final:} \href{https://www.python.org/}{\textit{Python 3}}.

En un principio mi intención era tomar esta elección dándole bastante importancia a lo cómodo que me sentiría con un lenguaje en cuanto a gustos personales, ya que prefiero lenguajes tipados y más explícitos. Sin embargo, cuanto más avanzaba en las fases de diseño, más peso daba al sistema de recomendación. 

Es evidente que necesitaría algún tipo de API para servir los datos, pero por otra parte también necesitaría buenas herramientas de tratamiento y análisis de datos para poder construir un buen algoritmo de recomendaciones automático. Esto me llevó a pensar que probablemente el lenguaje que buscaba fuera bastante popular, pues habría tener buenas bibliotecas de desarrollo de APIs y de tratamiento de datos, las cuales suelen estar relacionadas con una buena comunidad de desarrolladores.

La decisión, como se indica arriba, fue Python. Este lenguaje he de admitir que no se adapta mucho a mis gustos personales en cuanto a sintaxis, pero lo que ofrece me será de gran utilidad para este desarrollo.

\textbf{Alternativa:} \href{https://go.dev/}{\textit{Go}}.

En cuanto a lenguajes existe mucha variedad, por la que la cantidad de alternativas que puede haber a Python, muchas de ellas totalmente viables a día de hoy, es considerable. Es posible que mi elección no haya sido la más acertada ya que me preocupa el rendimiento, pues en una asignatura del Master lo usé para construir un algoritmo genético y los tiempos de ejecución resultaron ser devastadores.

Es cierto que la velocidad del lenguaje \cite{languagespeed} es importante, pero también lo son las utilidades que ofrece y el nivel de mantenimiento y actualizaciones que prolonga su ciclo de vida.

Me he decantado por poner esta alternativa ya que es la que tenía en mente al comienzo del proyecto. Sintácticamente es un lenguaje que me encanta, además del control de punteros explícito y otras características, y ya lo había probado anteriormente para desarrollar una API REST y me gustó bastante. Además, tiene al equipo de Google detrás, por lo que inspira bastante confianza.

Demuestra ser bastante más rápido que Python en \href{https://benchmarksgame-team.pages.debian.net/benchmarksgame/fastest/go-python3.html}{bastantes situaciones} \cite{languagespeed}, pero por desgracia no encontré librerías prometedoras con utilidades para el tratamiento de datos que me permitieran, entre otras cosas, establecer correlaciones.

\subsubsection{\textit{Test framework.}}

\textbf{Elección final:} \href{https://docs.Pytest.org/}{\textit{Pytest}}

Existen muchos marcos de test diferentes, cada uno de ellos con fortalezas y debilidades además de otras características que aportan atractivo según la metodología y mentalidad con la que se enfoque el desarrollo.

En este caso se ha escogido \textit{Pytest} por la sencillez de su uso, ya que el único requisito que hay que aplicar en el código es prefijar el nombre de ficheros, clases y funciones con \textit{test}. Este prefijo es el que el \textit{framework}, a través de la orden \textit{Pytest}, buscará para determinar qué debe ejecutar, y el resultado de dichas ejecuciones determinará el éxito o fallo del test.

\textbf{Alternativas:}

Una opción muy popular es \href{https://robotframework.org/}{Robot Framework}, mediante el cual el programador define procesos y les otorga nombres empleando \textit{keywords} en lenguaje natural, lo que le permite construir tests para funcionalidades más complejas ordenando estas \textit{keywords}, dando lugar a una lista de pasos en lenguaje natural que definen una funcionalidad completa.

Como salida ofrece, además del resultado en la línea de comandos, archivos en formato \textit{html} con la información en un formato gráfico más visual y organizado.

Todo esto, a primera vista, suena muy llamativo, pero implica una carga extra de trabajo y un aumento de la deuda técnica que, en este caso, no me resultaba nada de interés, por lo que la descarté.

\href{https://docs.python.org/3/library/unittest.html}{\textit{Python unittest}} es otra alternativa, muy similar a \textit{Pytest}, cuya diferencia principalmente es que requiere la importación de un módulo y los tests definidos deben ser estrictamente clases que hereden de una en el módulo importado, dentro de la cual se deben definir las funciones de \textit{testing}. Como la única diferencia notable es tener que escribir un poco más de código y no aporta más beneficios, \textit{unittest} quedó descartada.

\subsubsection{Herramientas ajenas a la programación.}

\textbf{Para escribir la documentación:} \href{https://www.latex-project.org/}{\textit{LaTeX}} junto a \href{https://www.markdownguide.org/getting-started/}{\textit{Markdown}}.

Aunque LaTeX es algo más tedioso de usar puesto que la configuración inicial es mucho más compleja que cualquier procesador de textos, es una opción muy potente a la hora de construir documentos bien organizados. Si tuviera que elegir una alternativa, probablemente sería \href{https://workspace.google.com/intl/es/products/docs/}{Documentos de Google}, no es la más potente en cuanto a características de formato comparado con \href{https://www.microsoft.com/es-es/microsoft-365/word}{Microsoft Word} o su contraparte libre \href{https://es.libreoffice.org/descubre/writer/}{Writer}, pero está en la nube de forma nativa, lo cual resulta muy cómodo para trabajar desde cualquier lugar.

Por otra parte está \textit{Markdown}. Se trata de un lenguaje de marcas para dar formato al texto muy popular, el cual se encuentra cada vez en más sitios web a disposición de sus usuarios. En el caso de este proyecto, utilizo \textit{Markdown} para toda la documentación que quiero que sea visible directamente en el \href{https://github.com/Anglepi/My-Many-Reads}{repositorio del proyecto}. Dado que no existen alternativas que estén tan extendidas, realmente no tiene competencia para su uso en sitios web.

\textbf{Revisión ortográfica:} \href{http://aspell.net/}{\textit{aspell}}.

Es un corrector ortográfico \textit{Open Source} que empleo para la corrección de mi documentación. El principal motivo de la elección de ésta es que, a parte de ser \textit{Open Source}, tiene soporte para ficheros de LaTeX. En el repositorio hay \href{https://github.com/Anglepi/My-Many-Reads/blob/main/scripts/spellcheck.sh}{un \textit{script}} que se encarga de su uso, apoyándose en un diccionario personalizado de \href{https://github.com/Anglepi/My-Many-Reads/blob/main/scripts/spellcheckDictionary.txt}{excepciones} ya que esta comprobación ortográfica es parte del proceso de integración, y si no supera esta comprobación no se permitirá la inclusión de los cambios a la rama principal.

La alternativa más potente es \href{https://www.antidote.info/en/}{\textit{Antidote} de \textit{Druide}}, que además ofrece comprobaciones gramaticales las cuales parecen ser bastante acertadas. Sin embargo, al ser una opción de pago, se ha descartado.

\subsection{Módulos importados}

\textbf{Biblioteca de aserciones:} \href{https://github.com/assertpy/assertpy}{\textit{assertpy}}

Una biblioteca con varias funciones de aserciones para encapsular todo el proceso de \textit{assert} junto con las comprobaciones pertinentes. Partiendo de la función \textit{assert\_that} y encadenando otras, permite construir aserciones en lenguaje casi natural.

La alternativa más prometedora que consideré fue \href{https://github.com/grappa-py/grappa}{\textit{Grappa}}, con la cual también construyes las aserciones formando frases en lenguaje natural encadenando funciones. No hay mucha diferencia con assertpy en cuanto a características ofrecidas, así que por gustos personales me decanté por la otra opción.