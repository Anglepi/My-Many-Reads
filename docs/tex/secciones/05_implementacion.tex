\chapter{Implementación}

En este capítulo se describe el proceso de implementación junto con todos los elementos en los que éste se apoya, una descripción de todas las elecciones técnicas tomadas junto con su justificación.

En primer lugar se muestra el listado de las historias de usuario generadas junto con sus correspondientes tareas, que dan forma a las unidades de trabajo refinadas. Tras esto, se encuentran todos los dilemas tecnológicos considerados, junto con la solución seleccionada y  una lista de alternativas brevemente definidas con el fin de proveer una justificación fundada y válida de la elección tomada.

\subsection{Historias de usuario}

Partiendo de los requisitos previamente listados, refinándolos en historias de usuario, se puede dar paso a la elaboración de tareas que guíen la implementación del proyecto. Cabe aclarar que, como este desarrollo sigue una mentalidad ágil, las historias y tareas se han ido generando a corto plazo, por lo que la siguiente lista de historias de usuario y sus tareas se ha ido construyendo a lo largo del proceso de desarrollo.

Algunas historias, como ya se verá a continuación, aparecen repetidas. Esto se debe a que su desarrollo ha abarcado más de un PMV, pero en cada uno de estos se tratan diferentes tareas, por lo que solo se citaran las que correspondan al PMV descrito.

Así mismo cabe mencionar que el primer \textit{milestone} del proyecto se excluye de esta sección, pues está compuesto únicamente de documentación no directamente relacionada con las tareas de implementación.

\subsubsection{PMV-1. Estructura de datos básica y base de la lógica de negocio}

\begin{itemize}
    \item \href{https://github.com/Anglepi/My-Many-Reads/issues/7}{\textbf{HU-1}} Criterios de evaluación del jurado \\
    Como miembro del jurado, \\
    quiero recibir una documentación y presentación sobre este proyecto y su desarrollo, \\
    con el fin de poder evaluarlo en base a unos criterios específicos.

    \item \href{https://github.com/Anglepi/My-Many-Reads/issues/29}{\textbf{HU-2}} Solicitar información de libros \\
    Como Sergio, lector en MMR, \\
    quiero solicitar información sobre libros de una base de datos, \\
    con el fin de saber si me pueden interesar según sus características.
    \begin{itemize}
        \item \href{https://github.com/Anglepi/My-Many-Reads/issues/31}{\textbf{T-2.1}} Definir estructura básica de datos para los libros.
        \item \href{https://github.com/Anglepi/My-Many-Reads/issues/32}{\textbf{T-2.2}} Crear primera aproximación de BD de libros.
        \item \href{https://github.com/Anglepi/My-Many-Reads/issues/33}{\textbf{T-2.3}} Métodos para tratar las propiedades de los libros.
    \end{itemize}

    \item \href{https://github.com/Anglepi/My-Many-Reads/issues/30}{\textbf{HU-3}} Bibliotecas para gestionar lecturas
    Como Sergio, lector en MMR, \\
    quiero una biblioteca de libros de mi elección, \\
    con el fin de gestionar mis lecturas.
    \begin{itemize}
        \item \href{https://github.com/Anglepi/My-Many-Reads/issues/34}{\textbf{T-3.1}} Definir estructura básica de datos para las bibliotecas.
        \item \href{https://github.com/Anglepi/My-Many-Reads/issues/35}{\textbf{T-3.2}} Funcionalidad básica de gestión de bibliotecas.
    \end{itemize}
\end{itemize}