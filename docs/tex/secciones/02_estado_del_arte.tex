\chapter{Estado del arte}
\label{Estado del arte}

Con lo descrito hasta ahora, se pueden identificar fácilmente dos tipos de usuarios o clientes de \textit{My Many Reads}, aunque puede que en el futuro aparezcan más. Estos son:
\begin{itemize}
    \item El \textbf{usuario lector}, que visita la plataforma como un punto de interés y apoyo para su \textit{hobby}, generando información y estadísticas de uso en el proceso. \label{usuario lector}
    \item El \textbf{usuario consumidor de información}, que consume el tráfico de información generado con el fin de introducir anuncios o hacer estudios de mercado.
\end{itemize}

\section{Problemas de los usuarios}
\label{problemas de los usuarios}

Entre estos dos usuarios se puede generar la siguiente lista de problemas:
\begin{itemize}
    \item \textbf{Usuario lector}
    \begin{itemize}
        \item Necesita una herramienta para gestionar sus lecturas.
        \item Necesita recomendaciones de libros nuevos, que se adapten a sus gustos.
        \item Necesita explorar libremente los libros existentes por características como la popularidad, en caso de querer explorar nuevas opciones.
        \item Necesita conocer libros similares entre sí, junto a los motivos de la similitud, en caso de querer leer algo parecido a un libro que ya ha leído.
    \end{itemize}
    \item \textbf{Usuario consumidor de información}
    \begin{itemize}
        \item Necesita saber cuales son las tendencias más populares, como autores o géneros, para conocer el estado del mercado actual.
        \item Necesita encontrar la audiencia adecuada a la que publicitar libros.
    \end{itemize}
\end{itemize}

Actualmente solo destacan otras dos plataformas similares, Goodreads y Librarything, siendo la primera la opción más conocida y popular en cuanto a este tipo de plataformas. A continuación se exploran las características de estas dos plataformas, listando aspectos positivos y negativos.

\section{Plataformas existentes}

En esta sección se exploran las características de algunas plataformas ya existentes similares a \textit{My Many Reads}, resolviendo problemas comunes y compartiendo objetivos.

En sus correspondientes secciones se realiza un análisis exploratorio de ambas plataformas, enumerando aspectos positivos y negativos con el fin de descubrir posibles fortalezas y debilidades a tener en cuenta durante el desarrollo de \textit{My Many Reads}, y finalmente incluyendo una tabla en la que se indica qué problemas de los mencionados anteriormente solventa y hasta que punto.

\subsection[Goodreads]{Goodreads\\ {\large \url{https://www.goodreads.com/}}}

Sin duda la opción más popular, esta plataforma ofrece una serie de opciones bastante llamativas y útiles:

\begin{itemize}
    \item Una biblioteca de libros a la que puedes añadir libros leídos, en progreso y en lista de espera, junto con tu puntuación, fechas de inicio y fin de lectura y \textit{review}.
    \item Permite añadir libros por tu cuenta, en caso de que estos no se encuentren en la web de la página por ser ediciones antiguas, editoriales pequeñas o cualquier otro motivo.
    \item Un sistema de recomendaciones basado en tu biblioteca.
    \item Estadísticas de tus lecturas.
    \item Listas creadas por la comunidad que recogen libros con características comunes, como puede ser \textit{Books That Should Be Made Into Movies}.
    \item Un foro de discusiones, un apartado de citas famosas y secciones de AMA (\textit{Ask Me Anything}) dirigidas a autores.
    \item Una sección de noticias y artículos relacionados con los libros y el mundo de la lectura en general.
    \item Elementos de \textit{gamificación} de la lectura como preguntas diarias, un \textit{challenge} de lecturas anuales y cuestionarios acerca de libros.
    \item Permite a los usuarios escribir y compartir sus propias historias.
\end{itemize}

No por nada esta plataforma es la más popular, pero sin embargo no carece de aspectos mejorables:

\begin{itemize}
    \item La interfaz en sí es bastante mejorable, se hace difícil a un usuario promedio navegar y descubrir todo lo que ofrece sin tener que realizar un poco de esfuerzo.
    \item No permite explorar cómodamente la base de datos de libros, solo puedes ver las tendencias actuales (decididas por la plataforma mediante algún algoritmo desconocido) y explorar de un género concreto.
    \item Las listas mencionadas anteriormente las crea un único usuario, por lo que solo la opinión de una persona es la que dicta la relación entre todos los libros de la lista.
    \item El apartado de discusión de libros no ofrece mucha libertad, una vez lo encuentras parece más bien un conjunto de publicaciones filtrado según si son acerca de o relacionados con el libro concreto.
    \item Al visitar la página de un libro, si quieres consultar opiniones de la gente, lo que se te ofrece es su puntación media (no precisamente muy visible) y una sección de comentarios como la que te podrías encontrar en Amazon o cualquier tienda online. El foro de discusiones mencionado anteriormente es una fuente valiosa de \textit{feedback} que, aún con los problemas mencionados anteriormente, no parece ser accesible desde esta página.
    \item Información adicional como la existencia de adaptaciones en otros formatos (tales como películas o series) no aparece por ninguna parte.
    \item No se hace mención a diferentes ediciones de un libro ni a los idiomas en los que está disponible.
    \item El autor, otro elemento del cual un usuario podría usar para guiarse a la hora de buscar su siguiente lectura, solo dispone de un perfil muy similar al que podría tener cualquier usuario común, diferenciándose en una sección de AMA y de listado de sus libros. No parece haber un lugar en el que discutir sobre el estilo narrativo u otros elementos comunes a todos sus libros, que puedan llevar a un usuario a leer diferentes libros del autor que no compartan género entre sí.
\end{itemize}

En la tabla siguiente se muestra el grado de cumplimiento que tiene esta plataforma con respecto a los problemas listados anteriormente:

\begin{table}[H]
    \begin{tabularx}{\linewidth}{|X|X|}
        \hline
        \textbf{Problema} & \textbf{¿Lo solventa \textit{Goodreads}?} \\
        \hline
        \hline
        \multicolumn{2}{|l|}{\textbf{Usuario lector}} \\
        \hline
        Tiene una herramienta para gestionar lecturas     & \textbf{Sí}, con \textit{shelves} para categorizar tus libros \\ 
        \hline
        Ofrece recomendaciones adaptadas a los gustos del usuario & \textbf{Sí}, según géneros, \textit{shelves} y privadas de otros usuarios \\ 
        \hline
        Permite explorar libros estableciendo criterios de búsqueda   & \textbf{Parcialmente}, sólo por géneros \\ 
        \hline
        Permite consultar libros relacionados con otros   & \textbf{Parcialmente}, pero sin justificación \\ 
        \hline
        \multicolumn{2}{|l|}{ \textbf{Usuario consumidor de información}} \\
        \hline
        Muestra las tendencias actuales     & \textbf{Parcialmente}, muestra 25 tendencias entre todos los usuarios del sitio, además de unos premios anuales. No permite por ejemplo tendencias por género o autor. \\ 
        \hline
        Muestra publicidad dirigida     & \textbf{Sí}, a través de artículos, solo uno por libro. Para ver más hay que ir a una sección propia. Además, muestra enlaces de compra en tiendas online. \\ 
        \hline
    \end{tabularx}
\end{table}

\subsubsection{Modelo de negocio}

En esta plataforma, el modelo de negocio queda claramente visible. En el pie de página se puede observar una sección denominada \textit{Work with us} en la que aparecen enlaces de interés para autores y anunciantes que puedan estar interesados en darse a conocer o adquirir más público a través de la plataforma.

Por otra parte, y no quedando reflejado tan claro de cara al usuario, también venden información con el objetivo de mostrar publicidad más relevante a los usuarios, y los enlaces a tiendas online para comprar libros que muestra suelen tener incrustados algún código de referencia mediante el cual, probablemente, la página gane cierta comisión por las ventas realizadas a través de dicho enlace.

\subsection[Librarything]{Librarything\\ {\large \url{https://www.librarything.com/}}}

La opción menos popular, no da indicios de la funcionalidad al completo que ofrece hasta que el usuario se registra. Además de ofrecer menos funcionalidad que Goodreads, parece que tiene algunos problemas adicionales. Comenzando por lo que ofrece esta plataforma:

\begin{itemize}
\item Una biblioteca de libros para cada usuario, junto con un sistema de recomendaciones automáticas en base a esta biblioteca.
\item Una sección de comunidades, similares a grupos de lectura, en las que compartir opiniones, embarcarse en retos o hacer concursos.
\item Un foro de discusiones, similar al de Goodreads, pero más accesible.
\item Búsqueda de autores.
\item Conexión de tu perfil con otras aplicaciones, como redes sociales o plataformas de intercambio de libros.
\item Muestra más metadatos sobre los libros que Goodreads, como idiomas en los que está disponibles y portadas que puedes encontrar.
\item Permite consultar estadísticas de los usuarios de la plataforma, lo cual resulta de utilidad a la hora de decidir si involucrarse con esta plataforma.
\end{itemize}

La funcionalidad ofrecida es claramente menor, aunque cabe decir que tiene lo suficiente como para poder atraer a un conjunto de usuarios bastante amplio. Pero de nuevo, no todo se ofrece de la mejor forma posible, existen aspectos susceptibles de mejora:

\begin{itemize}
\item De nuevo, la interfaz resulta claramente mejorable, lo que puede causar pérdida de usuarios.
\item Se usan diferentes bibliotecas según el estado de lectura de un libro, y solo permite al usuario añadir una puntuación y etiquetas a cada libro añadido a la biblioteca.
\item No existe un buscador libros o base de datos propia, lo que permite agregar dos veces el mismo libro si se busca en páginas o ediciones diferentes, por lo que aparecen inconsistencias.
\item Tiene un buscador de librerías o bibliotecas cercanas a tu zona, lo cual no está mal, pero existen servicios similares más populares y fiables, como el propio Google Maps.
\item Como en Goodreads, solo permite explorar libros por género o etiqueta, sin ningún tipo de filtro más allá de eso.
\item No existen listas de libros y las recomendaciones escritas por usuarios no tienen por qué incluir el motivo de la recomendación, lo cual no genera mucha confianza.
\end{itemize}

En la tabla siguiente se muestra el grado de cumplimiento que tiene esta plataforma con respecto a los problemas listados anteriormente:

\begin{table}[H]
    \begin{tabularx}{\linewidth}{|X|X|}
        \hline
        \textbf{Problema} & \textbf{¿Lo solventa \textit{Librarything}?} \\
        \hline
        \hline
        \multicolumn{2}{|l|}{\textbf{Usuario lector}} \\
        \hline
        Tiene una herramienta para gestionar lecturas     & \textbf{Sí}, con colecciones para categorizar tus libros \\ 
        \hline
        Ofrece recomendaciones adaptadas a los gustos del usuario & \textbf{Sí}, de entre los más populares y los más recientes \\ 
        \hline
        Permite explorar libros estableciendo criterios de búsqueda   & \textbf{Parcialmente}, sólo por términos de popularidad \\ 
        \hline
        Permite consultar libros relacionados con otros   & \textbf{Parcialmente}, por autor, géneros, comunes a los usuarios, pero sin justificación \\ 
        \hline
        \multicolumn{2}{|l|}{ \textbf{Usuario consumidor de información}} \\
        \hline
        Muestra las tendencias actuales     & \textbf{Sí}, en intervalos de tiempo configurables, así como otras estadísticas sobre autores, series e idiomas. \\ 
        \hline
        Muestra publicidad dirigida     & \textbf{Parcialmente}, solo enlaces a tiendas online. \\ 
        \hline
    \end{tabularx}
\end{table}

\subsubsection{Modelo de negocio}

Hasta hace poco, parece ser que esta plataforma subsistía a base de suscripciones mensuales, hasta que debido a la pandemia del Covid-19, decidieron cambiar este aspecto y volverse gratuito.

A pesar de todo esto, siguen sin mostrar anuncios a los usuarios miembro, y su modelo de negocio no termina de estar totalmente claro. Es cierto que ofrecen estadísticas que podrían ser útiles, pero son de acceso libre. 

Teniendo en cuenta este último cambio, y el poco énfasis realizado en mostrar anuncios y publicitar nuevo contenido del mundo de la lectura, no aporta una sensación de tener algún tipo de beneficios desde el punto de vista del usuario promedio, pero en los términos de servicio se indica que esta plataforma se reserva los derechos de vender toda la información anónima, es decir, que no se pueda asociar a una persona concreta.
 
\subsection{Conclusiones}

Aunque también hay que mencionar que existen otras plataformas de este estilo pero con temática diferente, en cuanto a libros hay pocas opciones, y la que parece la mejor de ellas, Goodreads, tiene una amplitud de aspectos que se podrían mejorar, y por tanto introducirse en este entorno no parece una lucha imposible.

A continuación se muestran dos listas, una para enumerar todas las características que me han parecido muy buenas y que se encuentran presentes en alguna de estas plataformas, y otra para todos los aspectos negativos que considero que serían innecesarios o incluso perjudiciales para \textit{My Many Reads}.

Características más destacables:

\begin{itemize}
\item La posibilidad de crear una o varias colecciones (bibliotecas) de libros para recoger tus lecturas, con la posibilidad de añadir anotaciones para indicar el estado de la lectura u otras valoraciones.
\item Un sistema de recomendaciones automático en función de los libros incluidos en tu biblioteca.
\item Mantener una buena relación entre un libro y otros componentes relacionados con este, como información adicional (idiomas disponibles, página del autor, fecha de publicación, ediciones...) o contenido generado por otros usuarios (recomendaciones personalizadas, \textit{reviews}, puntuación...)
\item Proveer de herramientas a los usuarios para que su feedback (opiniones, puntuaciones, recomendaciones, etc.) quede reflejado y sea de utilidad para el resto de usuarios y los algoritmos de la plataforma.
\end{itemize}

Aspectos negativos:

\begin{itemize}
\item La interfaz no refleja correctamente la importancia de algunos elementos que no deberían pasar muy desapercibidos, como aquellos que sirven para indicar la popularidad de un libro.
\item No se ofrece mucha libertad a la hora de explorar libros. Sería interesante una sección dedicada a buscar libros según géneros, popularidad, puntuación, etc. configurando algún tipo de filtro más complejo.
\item Al consultar los libros relacionados con uno en concreto, no especifican por qué están relacionados con estos salvo por \textit{``A muchos usuarios también les gustó este otro libro``}. Más detalles del por qué están relacionados ayudarían a un usuario a tomar una decisión.
\item La abundancia de funcionalidades \textit{``Nice to have``} como los \textit{challenges}, \textit{quizzes}, etc. distraen al usuario de la funcionalidad principal.
\end{itemize}

\section{En \textit{My Many Reads}}

Considerando los puntos débiles y fuertes de estas plataformas (principalmente \textit{Goodreads} ya que es la más notable) se puede elaborar una lista de los elementos principales que resultarán de mayor utilidad para el inicio del ciclo de vida, ya que otros, como los elementos de gamificación, solo resultarían interesantes cuando se alcance cierta cantidad de usuarios activos en la plataforma, y por tanto no es necesario considerarlos de momento.

\begin{table}[H]
    \begin{tabularx}{\linewidth}{|X|X|}
        \hline
        \textbf{Apartado de interés} & \textbf{Propuesta de mejora} \\
        \hline
        Biblioteca gestionable     & La utilidad de mayor importancia en este tipo de aplicaciones. Tanto en \textit{Goodreads} como en \textit{Librarything} estas son similares, y difícilmente mejorables. Se propone mantener las características ofrecidas por estas con una interfaz mejorada y la opción adicional de incluir comentarios privados acerca de cada libro, que sirvan al lector como anotaciones para el futuro incluyendo aspectos negativos y positivos que sean de utilidad para decidir si volver a realizar la lectura. \\ 
        \hline
        Sistema de recomendaciones & Las recomendaciones a partir de la biblioteca solo se pueden mejorar con un ajuste del algoritmo. Dado que este es privado para cada plataforma, habrá que construir uno nuevo realizando un estudio de los parámetros a considerar y el peso de estos para el resultado. Adicionalmente se pretende ofrecer una nueva modalidad de recomendaciones en la que los usuarios relacionen dos libros justificando dicha relación, permitiendo a otros usuarios votar por esta y consultarlas por popularidad para obtener una información más relevante. \\ 
        \hline
        Motor de búsqueda   & En ambas plataformas el sistema de búsqueda de libros no permite una gran libertad a la hora de consultar libros en general en base a ciertos criterios de búsqueda, probablemente porque confíen en que solo se quiera buscar en base a las recomendaciones. Sin embargo, es posible que alguien quiera salir un poco de su zona de comfort de lectura y explorar nuevos géneros. Esto demuestra la utilidad de un sistema de búsqueda con capacidades más complejas de filtrado, combinando características como popularidad y géneros, autores o incluso fechas de publicación. \\ 
        \hline
    \end{tabularx}
\end{table}