\chapter{Estado del arte}

Actualmente solo destacan otras dos plataformas similares, Goodreads y Librarything, siendo la primera la opción más conocida y popular en cuanto a este tipo de plataformas. A continuación se exploran las características de estas dos plataformas, listando aspectos positivos y negativos.

\section[Goodreads]{Goodreads\\ {\large \url{https://www.goodreads.com/}}}

Sin duda la opción más popular, esta plataforma ofrece una serie de opciones bastante llamativas y útiles:

\begin{itemize}
    \item Una biblioteca de libros a la que puedes añadir libros leídos, en progreso y en lista de espera, junto con tu puntuación, fechas de inicio y fin de lectura y \textit{review}.
    \item Un sistema de recomendaciones basado en tu biblioteca.
    \item Estadísticas de tus lecturas.
    \item Listas creadas por la comunidad que recogen libros con características comunes, como puede ser \textit{Books That Should Be Made Into Movies}.
    \item Un foro de discusiones, un apartado de citas famosas y secciones de AMA (\textit{Ask Me Anything}) dirigidas a autores.
    \item Una sección de noticias y artículos relacionados con los libros y el mundo de la lectura en general.
    \item Elementos de \textit{gamificación} de la lectura como preguntas diarias, un \textit{challenge} de lecturas anuales y cuestionarios acerca de libros.
    \item Permite a los usuarios escribir y compartir sus propias historias.
\end{itemize}

No por nada esta plataforma es la más popular, pero sin embargo no carece de aspectos mejorables:

\begin{itemize}
    \item La interfaz en sí es bastante mejorable, se hace difícil a un usuario promedio navegar y descubrir todo lo que ofrece sin tener que realizar un poco de esfuerzo.
    \item No permite explorar cómodamente la base de datos de libros, solo puedes ver las tendencias actuales (decididas por la plataforma mediante algún algoritmo desconocido) y explorar de un género concreto.
    \item Las listas mencionadas anteriormente las crea un único usuario, por lo que solo la opinión de una persona es la que dicta la relación entre todos los libros de la lista.
    \item El apartado de discusión de libros no ofrece mucha libertad, una vez lo encuentras parece más bien un conjunto de publicaciones filtrado según si son acerca de o relacionados con el libro concreto.
    \item Al visitar la página de un libro, si quieres consultar opiniones de la gente, lo que se te ofrece es su puntación media (no precisamente muy visible) y una sección de comentarios como la que te podrías encontrar en Amazon o cualquier tienda online. El foro de discusiones mencionado anteriormente es una fuente valiosa de \textit{feedback} que, aún con los problemas mencionados anteriormente, no parece ser accesible desde esta página.
    \item Información adicional como la existencia de adaptaciones en otros formatos (tales como películas o series) no aparece por ninguna parte.
    \item No se hace mención a diferentes ediciones de un libro ni a los idiomas en los que está disponible.
    \item El autor, otro elemento del cual un usuario podría usar para guiarse a la hora de buscar su siguiente lectura, solo dispone de un perfil muy similar al que podría tener cualquier usuario común, diferenciándose en una sección de AMA y de listado de sus libros. No parece haber un lugar en el que discutir sobre el estilo narrativo u otros elementos comunes a todos sus libros, que puedan llevar a un usuario a leer diferentes libros del autor que no compartan género entre sí.
\end{itemize}

\section[Librarything]{Librarything\\ {\large \url{https://www.librarything.com/}}}

La opción menos popular, no da indicios de la funcionalidad al completo que ofrece hasta que el usuario se registra. Además de ofrecer menos funcionalidad que Goodreads, parece que tiene algunos problemas adicionales. Comenzando por lo que ofrece esta plataforma:

\begin{itemize}
\item Una biblioteca de libros para cada usuario, junto con un sistema de recomendaciones automáticas en base a esta biblioteca.
\item Una sección de comunidades, similares a grupos de lectura, en las que compartir opiniones, embarcarse en retos o hacer concursos.
\item Un foro de discusiones, similar al de Goodreads, pero más accesible.
\item Búsqueda de autores.
\item Conexión de tu perfil con otras aplicaciones, como redes sociales o plataformas de intercambio de libros.
\item Muestra más metadatos sobre los libros que Goodreads, como idiomas en los que está disponibles y portadas que puedes encontrar.
\item Permite consultar estadísticas de los usuarios de la plataforma, lo cual resulta de utilidad a la hora de decidir si involucrarse con esta plataforma.
\end{itemize}

La funcionalidad ofrecida es claramente menor, aunque cabe decir que tiene lo suficiente como para poder atraer a un conjunto de usuarios bastante amplio. Pero de nuevo, no todo se ofrece de la mejor forma posible, existen aspectos susceptibles de mejora:

\begin{itemize}
\item De nuevo, la interfaz resulta claramente mejorable, lo que puede causar pérdida de usuarios.
\item Se usan diferentes bibliotecas según el estado de lectura de un libro, y solo permite al usuario añadir una puntuación y etiquetas a cada libro añadido a la biblioteca.
\item No existe un buscador libros o base de datos propia, lo que permite agregar dos veces el mismo libro si se busca en páginas o ediciones diferentes, por lo que aparecen inconsistencias.
\item Tiene un buscador de librerías o bibliotecas cercanas a tu zona, lo cual no está mal, pero existen servicios similares más populares y fiables, como el propio Google Maps.
\item Como en Goodreads, solo permite explorar libros por género o etiqueta, sin ningún tipo de filtro más allá de eso.
\item No existen listas de libros y las recomendaciones escritas por usuarios no tienen por qué incluir el motivo de la recomendación, lo cual no genera mucha confianza.
\end{itemize}

\section{Conclusiones}

Aunque también hay que mencionar que existen otras plataformas de este estilo pero con temática diferente, en cuanto a libros hay pocas opciones, y la que parece la mejor de ellas, Goodreads, tiene una amplitud de aspectos que se podrían mejorar, y por tanto introducirse en este entorno no parece una lucha imposible.