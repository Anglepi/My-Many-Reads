\chapter{Análisis del problema}

\section{\textit{User personas} y sus historias}

Una buena técnica para realizar un análisis en profundidad del problema y obtener más detalles acerca de cuales son los puntos críticos de la aplicación y cómo atenderlos de manera que se garantice una buena experiencia por parte del usuario. Estas herramientas no solo ayudan a comprender mejor el problema, sino también a elaborar una lista de requisitos funcionales y no funcionales, que son de gran ayuda a la hora de diseñar e implementar cualquier proyecto.

\section{Aprovechamiento de la plataforma}
Si bien es cierto que ya existen algunas plataformas similares (ver \nameref{Estado del arte}), se puede ver que no son perfectas, y de alguna manera se podrían introducir mejoras en algún sentido que podrían aumentar el número de usuarios, por lo que no sería descabellado pensar en hacerles competencia y tener algo de éxito.

El primer paso sería lograr una comunidad activa, para lo cual hay que ofrecer incentivos a los usuarios para emplear la plataforma, pues una red social sin usuarios no serviría para nada. Pero también hay que tener en cuenta que ofrecer un mero servicio a los usuarios sin obtener algún tipo de beneficio con el que cubrir los gastos de mantenimiento no lo más adecuado, por lo que habría que explotar este tráfico de usuarios e información de alguna manera:

\begin{itemize}
    \item Sirviendo como lugar publicitario para libros de todo tipo, ya que en la plataforma se concentrarían usuarios con interés en la lectura, y por tanto, posibles clientes.
    \item Así mismo, se podrían recoger datos sobre los gustos e intereses de los usuarios para que las propias editoriales sepan qué géneros y autores están más de moda, ayudándoles así a embarcarse en proyectos con mayor beneficio potencial.
\end{itemize} 
