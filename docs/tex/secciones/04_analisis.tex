\chapter{Análisis del problema}

\section{\textit{User personas} y \textit{user journeys}}

Tanto los \textit{user personas}\cite{personas} como los \textit{user journeys}\cite{userjourney} son buenas técnicas para realizar un análisis en profundidad del problema y obtener más detalles acerca de cuales son los puntos críticos de la aplicación y cómo atenderlos de manera que se garantice una buena experiencia por parte del usuario. Estas herramientas no solo ayudan a comprender mejor el problema, sino también a elaborar una lista de requisitos funcionales y no funcionales, que son de gran ayuda a la hora de diseñar e implementar cualquier proyecto.

En primer lugar, una lista de \textit{personas} aportará una idea general de qué tipos de usuario podrían usar la plataforma solución planteada, además de, posiblemente, información adicional sobre problemas que pudieran existir y que radiquen en características concretas del usuario. A continuación se explorarán las historias de los usuarios a través de la plataforma, a través de \textit{user journeys}, con el fin de determinar qué soluciones desean encontrar y cómo pueden hacerlo, así como el nivel de satisfacción obtenido al realizar las acciones necesarias para conseguir su objetivo, influenciado por el proceso y el resultado. 

\subsection{\textit{Personas}}

\begin{table}[H]
    \centering
    \begin{tabularx}{\columnwidth}{|l|X|}
        \hline
        \textbf{Nombre y edad} & Sergio, 24 años. \\
        \hline
        \textbf{Rol} & Fan de la lectura (usuario lector). \\
        \hline
        \textbf{Ocupación} & Estudiante de Marketing Digital. \\
        \hline
        \textbf{Biografía} & Ya finalizando sus estudios, Sergio encuentra más tiempo que dedicar a la lectura. Disfruta tanto de leer novelas de fantasía y ciencia ficción como libros de historia. Sabe manejarse bien con el ordenador, aunque solo es experto en herramientas de comunicación y diseño, y no suele usar otro tipo de aplicaciones. Le gusta mantener una lista de objetivos a corto, medio y largo plazo ya que le ayuda a llevar una mejor calidad de vida. \\
        \hline
        \textbf{Temores} & Su mayor preocupación es que no aparezcan los libros de historia que le regaló su abuelo. Por otra parte, su principal preocupación es que la biblioteca no le permita diferenciar sus lecturas de historia de las novelas, ya que él lo considera dos tipos de \textit{hobby} diferentes. \\
        \hline
        \textbf{Necesidades} & Quiere una herramienta que gestione sus lecturas, permitiéndole diferenciar entre finalizadas, en progreso y pendientes con el fin de apoyar su lista de objetivos. Como es muy organizado, le gustaría que la herramienta fuese fácil de gestionar y versátil en caso de querer cambiar la organización de sus lecturas. \\
        \hline
    \end{tabularx}
\end{table}

\begin{table}[H]
    \centering
    \begin{tabularx}{\columnwidth}{|l|X|}
        \hline
        \textbf{Nombre y edad} & Ana, 32 años. \\
        \hline
        \textbf{Rol} & Fan de la lectura (usuario lector). \\
        \hline
        \textbf{Ocupación} & Talent recruiter en una compañía de IT. \\
        \hline
        \textbf{Biografía} & Desde su adolescencia Ana ha disfrutado muchísimo de la lectura, aficionándose a muchos géneros diferentes. De hecho en casa tiene una estantería llena de libros, la cual organiza de diferentes maneras continuamente, incluso por colores de la portada. Últimamente le resulta difícil encontrar un nuevo libro para leer, y se ha dado cuenta de que ha releído bastantes libros, por lo que quiere explorar nuevos libros, incluso los no disponibles en su idioma. \\
        \hline
        \textbf{Temores} & A Ana le preocupa que \textit{My Many Reads} no ofrezca buenas herramientas para descubrir nuevos libros. Piensa que la única manera de obtener recomendaciones de un sistema de esta naturaleza es a partir de una biblioteca, y que por tanto la plataforma no le ofrecerá opciones muy diferentes a otras herramientas del estilo. También quiere explorar libros con libertad en función de sus propios criterios, y le preocupa no tener mucha libertad a la hora de hacer esta búsqueda. \\
        \hline
        \textbf{Necesidades} & Su principal necesidad es encontrar nuevas lecturas. Para ello no quiere depender solo de recomendaciones en base a su biblioteca, si no que le gustaría disponer de herramientas para explorar manualmente libros en base a unos criterios de búsqueda específicos. También le gustaría encontrar recomendaciones en base a un libro concreto, y si están justificadas por otros usuarios, mejor. \\
        \hline
    \end{tabularx}
\end{table}

\begin{table}[H]
    \centering
    \begin{tabularx}{\columnwidth}{|l|X|}
        \hline
        \textbf{Nombre y edad} & Julio, 29 años. \\
        \hline
        \textbf{Rol} & Consumidor de información generada. \\
        \hline
        \textbf{Ocupación} & Product Manager del equipo de análisis de datos de una editorial. \\
        \hline
        \textbf{Biografía} & Julio comenzó a trabajar en su nuevo puesto hace poco. En este le han encargado encontrar información sobre el mercado actual del libro, para saber qué géneros y autores tienen más potencial de alcanzar un mayor nivel de ventas. Trabaja con muchos tipos de herramientas para obtener la información a analizar, y le gusta estar al día con las noticias de nuevas publicaciones. \\
        \hline
        \textbf{Temores} & La principal preocupación de Julio es hacer una inversión para obtener información de baja calidad, ya que esto aumenta su trabajo añadiendo requisitos de preprocesamiento de datos y, por si fuera poco, reduce la fiabilidad de estos, lo cual le puede llevar a tomar malas decisiones que afecten a su trabajo. Tampoco quiere datos procesados, ya que desconocerá las técnicas de procesamiento y tampoco serán fiables. \\
        \hline
        \textbf{Necesidades} & Quiere encontrar una buena fuente de datos sobre los libros más populares en la actualidad, para conocer los autores y géneros con más mercado y por tanto más posibilidades de buenas ventas. \\
        \hline
    \end{tabularx}
\end{table}

\subsection{\textit{User journeys}}

Para crear una biblioteca y añadir libros:

\begin{itemize}
    \item \textbf{Acciones}
    \begin{enumerate}
        \item Desde la página principal, acceder a la sección de \textit{Mis bibliotecas}.
        \item En la lista de bibliotecas, seleccionar la opción \textit{Crear nueva biblioteca}, indicando el nombre de ésta.
        \item Con la biblioteca ya creada, acceder a la página de algún libro a través del buscador u otros medios.
        \item Desde la página del libro, buscar la opción \textit{Añadir a la biblioteca...} junto a los detalles del libro.
        \item En el menú emergente, seleccionar la nueva biblioteca.
    \end{enumerate}
\item \textbf{Puntos de contacto}
    \begin{itemize}
        \item Página principal.
        \item Lista de libros de un usuario.
        \item (Opcional) Buscador.
        \item Página de un libro.
    \end{itemize}
\item \textbf{Pensamientos}
    \begin{itemize}
        \item La creación de la biblioteca es un proceso sencillo.
        \item Los libros se añaden desde su página, y se accede a ellos de manera sencilla a través de un buscador en la barra de navegación presente en todas las vistas, lo cual resulta cómodo.
    \end{itemize}
\end{itemize}

Para buscar recomendaciones, existen varias formas:
\begin{itemize}
    \item \textbf{Acciones}
    \begin{enumerate}
        \item \textbf{Opción A.} Desde la página principal, acceder a la sección de \textit{Mis bibliotecas} y seleccionar la biblioteca cuyas recomendaciones se quieren consultar.
        \item En la página de la biblioteca, seleccionar la opción \textit{Recomendaciones}. Aparecerán recomendaciones generadas automáticamente a partir de los libros de la biblioteca.
        \item \textbf{Opción B.} Desde la página de un libro, acceder a la sección de \textit{Recomendaciones}. Aparecerán relaciones entre dicho libro y otros, realizadas y justificadas por otros usuarios.
    \end{enumerate}
\item \textbf{Puntos de contacto}
    \begin{itemize}
        \item Página principal.
        \item Lista de libros de un usuario.
        \item Lista de recomendaciones de una biblioteca.
        \item Página de un libro.
        \item Página de recomendaciones de otros usuarios en base a un libro.
    \end{itemize}
\item \textbf{Pensamientos}
    \begin{itemize}
        \item Las recomendaciones son fácilmente accesibles, y el algoritmo ofrece buenos resultados.
        \item Puedo consultar recomendaciones en base a mi libro favorito realizadas por otros usuarios con una justificación que me permite intuir si la recomendación es en base a mis aspectos favoritos del libro.
    \end{itemize}
\end{itemize}

\section{Aprovechamiento de la plataforma}
Si bien es cierto que ya existen algunas plataformas similares (ver \underline{\nameref{Estado del arte}}), se puede ver que no son perfectas, y de alguna manera se podrían introducir mejoras en algún sentido que podrían aumentar el número de usuarios, por lo que no sería descabellado pensar en hacerles competencia y tener algo de éxito.

El primer paso sería lograr una comunidad activa, para lo cual hay que ofrecer incentivos a los usuarios para emplear la plataforma, pues una red social sin usuarios no serviría para nada. Pero también hay que tener en cuenta que ofrecer un mero servicio a los usuarios sin obtener algún tipo de beneficio con el que cubrir los gastos de mantenimiento no lo más adecuado, por lo que habría que explotar este tráfico de usuarios e información de alguna manera:

\begin{itemize}
    \item Sirviendo como lugar publicitario para libros de todo tipo, ya que en la plataforma se concentrarían usuarios con interés en la lectura, y por tanto, posibles clientes.
    \item Así mismo, se podrían recoger datos sobre los gustos e intereses de los usuarios para que las propias editoriales sepan qué géneros y autores están más de moda, ayudándoles así a embarcarse en proyectos con mayor beneficio potencial.
\end{itemize} 

\section{Requisitos}

Con el apoyo de la información obtenida en las anteriores secciones se pueden definir unos requisitos que expresen las funcionalidades necesarias, los comportamientos esperados y los datos involucrados. Más adelante se elaborarán las diferentes historias de usuario para el desarrollo, enfocadas en cubrir todos los requisitos de esta lista.

A cada PMV planteado le corresponde un conjunto de requisitos, y dado que los PMV son incrementales, se espera que cada nuevo PMV cumpla los requisitos de todos los anteriores.

\subsection{\href{https://github.com/Anglepi/My-Many-Reads/milestone/2}{PMV-1}. Estructura de datos básica y base de la lógica de negocio}

\subsubsection{Requisitos de datos}
\begin{itemize}
    \item RD1-1 Almacenar libros con información básica acerca de éstos.
    \item RD1-2 Almacenar bibliotecas, asociadas a sus dueños, formadas por una lista de entradas:
    \begin{itemize}
        \item RD1-2.1 Cada entrada estará formada por un libro, su estado y la puntuación ofrecida por el usuario.
    \end{itemize}
\end{itemize}

\subsubsection{Requisitos funcionales}
\begin{itemize}
    \item RF1-1 El sistema debe permitir la creación de bibliotecas.
    \item RF1-2 El sistema debe permitir entradas a una biblioteca, permitiendo libros duplicados.
    \item RF1-3 Se debe permitir consultar los libros y bibliotecas almacenadas.
\end{itemize}

En este primer PMV se desarrollarán las clases necesarias para representar las estructuras de datos que definen los libros, bibliotecas y sus entradas, al menos un modelo básico con el que poder ofrecer una funcionalidad mínima en este punto.

Las funciones básicas ofrecidas en este punto serán las de creación de una biblioteca, añadir entradas a esta y consultar información existente sobre libros y bibliotecas. Una peculiaridad en este punto es que se permite añadir múltiples veces el mismo libro a la misma biblioteca. Esto se debe a que controlar esta situación requiere de lógica adicional que no se ha planificado para este PMV principalmente por dos motivos: más adelante podrían surgir cambios en el modelo de datos o la funcionalidad deseada que alteren la complejidad y por tanto requieran de trabajo adicional para refactorizar esta lógica (en otras palabras, existe el riesgo de que esto se convierta en deuda técnica), y por otra parte se trata de un PMV, como su nombre indica debe incluir el mínimo de detalles que lo hagan viable.

En los futuros PMV se trabaja no solo en añadir nuevos componentes al sistema, sino además en extender la funcionalidad de los anteriormente existentes, por lo que no debe ser motivo de preocupación dejar ``cabos sueltos'' que atar en el futuro.

\subsection{\href{https://github.com/Anglepi/My-Many-Reads/milestone/3}{PMV-2}. Estructura de datos básica y base de la lógica de negocio}

\subsubsection{Requisitos de datos}
\begin{itemize}
    \item RD2-1 Almacenar relaciones entre dos libros.
    \begin{itemize}
        \item RD2-1.1 Almacenar comentarios de usuario y sus valoraciones asociados a una relación entre dos libros.
    \end{itemize}
\end{itemize}

\subsubsection{Requisitos funcionales}
\begin{itemize}
    \item RF2-1 El sistema debe permitir realizar operaciones de forma remota.
    \item RF2-2 Se debe permitir a los usuarios crear relaciones entre dos libros a modo de recomendación.
\end{itemize}

El objetivo de este segundo PMV es, en pocas palabras, introducir una API que publique las funcionalidades implementadas al exterior, de forma que estas sean accesibles de forma remota. El hecho de incluir un nuevo método de acceso a la funcionalidad podría no considerarse un incremento de esta, por lo que además se incluyen las recomendaciones entre dos libros generadas por usuarios, una pequeña adición que implica nuevos modelos de datos sencillos, y que aporta un componente de interés a la plataforma.

Incluir la API es un paso importante a nivel arquitectónico, ya que ésta hará las veces de punto de contacto entre usuarios y/o sistemas externos con las funcionalidades de \textit{My Many Reads}, lo cual implica que cualquier funcionalidad incluida en la plataforma debe ser accesible directa o indirectamente desde la API.

\subsection{\href{https://github.com/Anglepi/My-Many-Reads/milestone/5}{PMV-3}. Base de datos remota}

\subsubsection{Requisitos funcionales}
\begin{itemize}
    \item RF3-1 El sistema debe ser capaz de realizar una conexión a una base de datos remota.
    \item RF3-2 Los cambios producidos por las transacciones de datos deben ser persistentes.
\end{itemize}

\subsubsection{Requisitos no funcionales}
\begin{itemize}
    \item RNF3-1 Los datos almacenados deben cumplir la legislación siguiendo las prácticas establecidas en esta.
\end{itemize}

Hasta este punto, \textit{My Many Reads} no empleaba una base de datos real, sino que almacenaba la información única y exclusivamente en entorno de ejecución. Esto, obviamente, representa una serie de problemas importantes, como una limitación grande en cuanto a cantidad de información o la imposibilidad de mantener dicha información entre diferentes ejecuciones (es decir, ausencia de persistencia).

Como los requisitos indican, en este PMV se van a incluir los cambios necesarios para que el sistema haga uso de una base de datos real. Esto implica una investigación sobre qué tipo de base de datos emplear, su compatibilidad con las tecnologías incluidas hasta este punto e introducir la lógica necesaria para sacarle partido.

A partir de este momento se gestionará la información de manera apropiada, en una base de datos que aporta persistencia y con una lógica que permite consistencia y transacciones coherentes de información. A nivel funcional esto sería lo mínimo necesario, sin embargo, quedan otros detalles que cubrir. Los datos almacenados deben cumplir las legislaciones pertinentes, en función de qué tipo de datos se almacenen. Esto por lo general suele implicar que la información introducida por un usuario deba ser eliminada si este así lo desea.

Esto supone un problema, puesto que toda esta información resulta de especial utilidad para la plataforma. Por una parte, dado su carácter social, tendría un impacto negativo la sensación de ausencia o decrecimiento de actividad de otros usuarios, pero por otro lado (y puede que más importante), el sistema de recomendaciones se vería afectado negativamente dado que depende, en parte, de la información generada por usuarios para generar mejores recomendaciones.

Para solventar esta problemática, se propone que la tarea de eliminar esta información suponga, en casos pertinentes, desvincularla totalmente de los usuarios creadores con el fin de que dicha información siga aportando utilidad a la plataforma. Obviamente, en el caso de datos de carácter personal si los hubiere, esta información será eliminada totalmente.

\subsection{\href{https://github.com/Anglepi/My-Many-Reads/milestone/4}{PMV-4}. Recolectar y ofrecer estadísticas}

\subsubsection{Requisitos funcionales}
\begin{itemize}
    \item RF4-1 El sistema debe permitir la consulta de visitas y nota media de libros.
    \item RF4-2 El sistema debe permitir la consulta de popularidad de géneros.
    \item RF4-3 El sistema debe permitir consultar un \textit{top} de libros según sus visitas o su nota media.
\end{itemize}

\subsubsection{Requisitos de datos}
\begin{itemize}
    \item RD4-1 Se debe almacenar el número de visitas realizadas a cada libro.
\end{itemize}

El proyecto debe tener algún elemento que le permita mantenerse. En este caso, se trata de información y estadísticas de uso, que se puede emplear en fines comerciales al visibilizar el estado del mercado actual indicando las tendencias más populares.

En este punto el sistema ya cuenta con las puntuaciones que los usuarios establecen en los libros a través de sus bibliotecas, por lo tanto el siguiente paso es agrupar todas estas puntuaciones para obtener una media referente a cada libro. Esto es algo que mediante consultas se puede conseguir, por lo que no se requieren cambios en la estructura de la base de datos.

Por otra parte se encuentra la popularidad, que consiste en el número de visitas que recibe cada libro. Esta información no se encuentra en la base de datos, por lo que habrá que editar la estructura de esta para incluirla. Adicionalmente, habrá que incluir alguna funcionalidad que incremente el número de visitas de un libro con cada consulta. Esto se puede lograr, dependiendo la base de datos, con operaciones o triggers definidos en ésta, lo cual facilita la lógica de uso de este proceso, o mediante la redefinición de la consulta empleada para solicitar los datos de un libro, logrando que en este momento también se incrementen las visitas. Dada la volatilidad experimentada en el sistema externo de base de datos, que se comentará en la sección de implementación. La segunda opción parece más sensata.

Una vez obtenido cierto control sobre la información deseada, hay que ofrecer la posibilidad de consultarla en formatos que puedan resultar interesantes. Una selección de formatos interesantes incluye, además de la más básica consulta de un libro mostrando su puntuación media y visitas, un \textit{top} de géneros más populares y un \textit{top} de libros según visitas o popularidad.

Este PMV está establecido principalmente como un primer paso hacia la satisfacción de los objetivos propuestos para los \href{https://github.com/Anglepi/My-Many-Reads/blob/main/docs/md/personas/information-consumer.md}{usuarios consumidores de información generada}.