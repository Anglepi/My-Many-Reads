\chapter{Análisis del problema}

\section{Aprovechamiento de la plataforma}
Si bien es cierto que ya existe algunas plataformas similares (ver \nameref{Estado del arte}), se puede ver que no son perfectas, y de alguna manera se podrían introducir mejoras en algún sentido que podrían aumentar el número de usuarios, por lo que no sería descabellado pensar en hacerles competencia y tener algo de éxito.

El primer paso sería lograr una comunidad activa, para lo cual hay que ofrecer incentivos a los usuarios para emplear la plataforma, pues una red social sin usuarios no serviría para nada. Pero también hay que tener en cuenta que ofrecer un mero servicio a los usuarios sin obtener algún tipo de beneficio con el que cubrir los gastos de mantenimiento no lo más adecuado, por lo que habría que explotar este tráfico de usuarios e información de alguna manera:

\begin{itemize}
    \item Sirviendo como lugar publicitario para libros de todo tipo, ya que en la plataforma se concentrarían usuarios con interés en la lectura, y por tanto, posibles clientes.
    \item Así mismo, se podrían recoger datos sobre los gustos e intereses de los usuarios para que las propias editoriales sepan qué géneros y autores están más de moda, ayudándoles así a embarcarse en proyectos con mayor beneficio potencial.
\end{itemize} 

\section{Requisitos}

Con el apoyo de la información obtenida en las anteriores secciones se pueden definir unos requisitos que expresen las funcionalidades necesarias, los comportamientos esperados y los datos involucrados. Más adelante se elaborarán las diferentes historias de usuario para el desarrollo, enfocadas en cubrir todos los requisitos de esta lista.

A cada PMV planteado le corresponde un conjunto de requisitos, y dado que los PMV son incrementales, se espera que cada nuevo PMV cumpla los requisitos de todos los anteriores.

\subsection{PMV-1. Estructura de datos básica y base de la lógica de negocio}

