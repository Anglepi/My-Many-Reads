\chapter{Introducción}

No se puede negar que internet es actualmente uno de los medios de comunicación más efectivos (si no el que más), ya que permite una comunicación rápida y de alcance global entre todos sus usuarios con una gran libertad.
\\\\
Para explotar al máximo estas características se han ido creando diversas plataformas para facilitar la comunicación entre usuarios, junto a las cuales han surgido diversas ideas sobre como explotar ese tráfico de usuarios y generar beneficios a la vez que se prestan servicios aparentemente gratuitos.
\\\\
\section{Descripción del problema}

A día de hoy, el afán por la lectura está aumentando entre no sólo los jóvenes (el sector que más preocupaba) sino adultos también. Una de las posibles causas de este hecho podría haber sido la pandemia, ya que con la cuarentena muchos han tenido que encontrar nuevos \textit{hobbies} a los que dedicarse desde su hogar para reemplazar actividades al aire libre.
\\\\
El problema es la gran diversidad de géneros y gustos personales a tener en cuenta a la hora de realizar una buena elección de un libro. En el caso de no tener contactos que puedan aconsejar, habría que recurrir a otras fuentes de información, principalmente internet ya que es la más accesible, amplia y fácil de utilizar de las disponibles.
\\\\
Se pueden encontrar ideas o sugerencias con simples búsquedas a través de Google, pero éstas suelen venir de revistas o blogs cuyos motivos de dichas recomendaciones no son transparentes, pudiendo ser, por ejemplo, totalmente personales del autor (una única persona) o haberse visto influenciados por un movimiento de patrocinio por parte de alguna editorial.
\\\\
Esto no parece una solución ideal, pues un usuario promedio fácilmente puede desconfiar de estas recomendaciones, ya que por una parte se desconoce de dónde proceden y cuales son los motivos detrás de éstas, y por otra parte podrían no encajar con los gustos del usuario.
\\\\
El propósito de este proyecto es acabar con este problema, permitiendo a todo aquel usuario con interés en la lectura servirse de unas recomendaciones totalmente personalizadas, apoyándose por opiniones que una multitud de usuarios similares aportan para cada libro. En otras palabras, elaborando las recomendaciones para los usuarios en base a sus lecturas y respaldando esas recomendaciones con opiniones de otros usuarios reales, por lo que se acaba creando una plataforma social en torno a la lectura en la que se refleja la opinión mayoritaria de sus usuarios, aportando mucha fiabilidad.

\section{Objetivos}
En el caso de \textit{My Many Reads}, el objetivo es construir una plataforma social en la que crear una comunidad de usuarios alrededor de la lectura. Partiendo de una base de datos de libros, se pretende reunir a una comunidad de usuarios con intereses en la lectura, ofreciéndoles una serie de servicios y funcionalidades como pueden ser:
\\\\
\begin{itemize}
    \item Gestión de una biblioteca personal, en la que los usuarios puedan almacenar los libros leídos, pendientes y abandonados junto a una valoración.
    \item Sistema de recomendaciones automático construyendo los gustos de los usuarios a partir de sus bibliotecas personales.
    \item Sistema de búsqueda de libros por sus características propias o popularidad en la plataforma.
    \item Foro de discusiones para cada libro en el que plasmar opiniones, teorías u otros pensamientos relacionados con dicho libro.
    \item Posibilidad de escribir recomendaciones personalizadas entre dos libros similares, que puedan ayudar a otros usuarios a tomar una decisión con su próxima lectura.
\end{itemize}

La meta global por tanto es la de construir una plataforma con características atractivas para atraer a un conjunto de usuarios. Sabiendo esto, el siguiente paso es establecer una lista de objetivos con el fin de planificar y diseñar en torno al cumplimiento de estos, enfocando todo el desarrollo en una dirección. En el caso de \textit{My Many Reads}, estos son los objetivos:

\begin{itemize}
    \item Cumplir con una funcionalidad mínima de consulta de libros y gestión listas o bibliotecas.
    \item Ofrecer una funcionalidad de recomendaciones personalizadas en función de los gustos de cada usuario.
    \item Permitir a los usuarios participar activamente en creación de contenido de la plataforma mediante foros de discusiones, recomendaciones entre libros personalizadas y reviews.
    \item Servir como medio de comunicación para ofrecer noticias pertenecientes al mundo de la lectura.
    \item Recoger estadísticas de los usuarios de la plataforma para conocer información sobre, por ejemplo, las tendencias actuales.
\end{itemize}

Estos dos últimos objetivos tienen cierta dependencia a que el proyecto se encuentre desplegado y con una base mínima de usuarios. Para llegar a ese punto, es imprescindible cumplir los objetivos previos cuya intención es atraer usuarios a la plataforma.
