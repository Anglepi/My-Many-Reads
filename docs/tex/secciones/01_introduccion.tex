\chapter{Introducción}

No se puede negar que internet es actualmente uno de los medios de comunicación más efectivos (si no el que más), ya que permite una comunicación rápida y de alcance global entre todos sus usuarios con una gran libertad.
\\\\
Para explotar al máximo estas características se han ido creando diversas plataformas para facilitar la comunicación entre usuarios, junto a las cuales han surgido diversas ideas sobre como explotar ese tráfico de usuarios y generar beneficios a la vez que se prestan servicios aparentemente gratuitos.
\\\\
En el caso de \textit{My Good Reads}, el objetivo es construir una plataforma social en la que crear una comunidad de usuarios alrededor de la lectura. Partiendo de una base de datos de libros, se pretende reunir a una comunidad de usuarios con intereses en la lectura, ofreciéndoles una serie de servicios y funcionalidades como pueden ser:
\\\\
\begin{itemize}
    \item Gestión de una biblioteca personal, en la que los usuarios puedan almacenar los libros leídos, pendientes y abandonados junto a una valoración.
    \item Sistema de recomendaciones automático construyendo los gustos de los usuarios a partir de sus bibliotecas personales.
    \item Sistema de búsqueda de libros por sus características propias o popularidad en la plataforma.
    \item Foro de discusiones para cada libro en el que plasmar opiniones, teorías u otros pensamientos relacionados con dicho libro.
    \item Posibilidad de escribir recomendaciones personalizadas entre dos libros similares, que puedan ayudar a otros usuarios a tomar una decisión con su próxima lectura.
\end{itemize}
