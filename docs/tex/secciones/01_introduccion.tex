\chapter{Introducción}

A día de hoy, el afán por la lectura está aumentando entre no sólo los jóvenes (el sector que más preocupaba) sino adultos también. Una de las posibles causas de este hecho podría haber sido la pandemia, ya que con la cuarentena muchos han tenido que encontrar nuevos \textit{hobbies} a los que dedicarse desde su hogar para reemplazar actividades al aire libre.

Como con cualquier otro tipo de afición, no es sencillo iniciarse en el mundo de la lectura sin conocer un buen punto de partida, pues lo ideal es comenzar con algo fácilmente disfrutable, lo cual depende fuertemente de los gustos personales de la persona en concreto, por lo que en general se suele hacer una pequeña investigación sobre aquellos que podrían ser interesantes.

Este tipo de búsquedas no solo las realizan aquellas personas introduciéndose en la lectura, sino también los ya acostumbrados lectores que acaban con su lista de libros pendientes o prefieren descubrir libros nuevos. Al fin y al cabo, cualquier persona interesada en la lectura acaba buscando algo que leer por los medios de que disponga.

\section{Motivación}

Justo antes de la pandemia decidí explorar nuevos \textit{hobbies} que me separaran de la pantalla. Compré varios juegos de mesa, empecé a pintar miniaturas y decidí comprarme un libro con una premisa medianamente interesante. De estos tres, las miniaturas fueron las únicas que acabé descartando, porque aunque me gustan, son bastante caras, y como para los juegos de mesa dependes de organizarte con tus amigos, me enfoqué bastante en la lectura.

Me encontré con el problema de que no conocía a mucha gente que leyera, y además, que compartiera mis gustos, por lo que dependía de buscar en internet en base a lo poco que llevaba leído, y la verdad es que era una tarea algo tediosa. Se me ocurrió que podría existir una plataforma de \textit{reviews} de la que podría servirme, y me topé con \textit{Goodreads}, pero la verdad es que el sitio no terminaba de gustarme del todo. Conozco otras plataformas para series y películas que sí son bastante cómodas de usar para estas búsquedas, pero no para libros.

Finalmente me acabé acostumbrando bastante a mi método "manual" de buscar nuevos libros, introduje a un par de amigos con mis mismos gustos a la lectura para discutir con ellos y acabé disfrutando más del \textit{hobby} en general. Sin embargo, siempre pensé (y sigo pensando) que todo esto habría sido más fácil e incluso mejor, si hubiera una plataforma online en la que explorar libros, compartir opiniones y recomendaciones, y gestionar las lecturas, en cierto modo parecido a lo que ofrece \textit{Goodreads} pero eliminando todos los detalles, aspectos y funcionalidades que hacen que yo, a pesar de disfrutar mucho de la lectura y de llevarme bien con los ordenadores en general, no quiera usar esta plataforma que parece estar pensada para gente con mi misma afición, y que además es la más destacable de este ámbito con diferencia.

\section{Descripción del problema}

Existe una gran diversidad de géneros y gustos personales a tener en cuenta a la hora de realizar una buena elección de un libro. En el caso de no tener contactos que puedan aconsejar, habría que recurrir a otras fuentes de información, principalmente internet ya que es la más accesible, amplia y fácil de utilizar de las disponibles.

Se pueden encontrar ideas o sugerencias con simples búsquedas a través de Google, pero éstas suelen venir de revistas o blogs cuyos motivos de dichas recomendaciones no son transparentes, pudiendo ser, por ejemplo, totalmente personales del autor (una única persona) o haberse visto influenciados por un movimiento de patrocinio por parte de alguna editorial.

Esto no parece una solución ideal, pues un usuario promedio fácilmente puede desconfiar de estas recomendaciones, ya que por una parte se desconoce de dónde proceden y cuales son los motivos detrás de éstas, y por otra parte podrían no encajar con los gustos del usuario.

El propósito de este proyecto es acabar con este problema, permitiendo a todo aquel usuario con interés en la lectura servirse de unas recomendaciones totalmente personalizadas, apoyándose por opiniones que una multitud de usuarios similares aportan para cada libro. En otras palabras, elaborando las recomendaciones para los usuarios en base a sus lecturas y respaldando esas recomendaciones con opiniones de otros usuarios reales, por lo que se acaba creando una plataforma social en torno a la lectura en la que se refleja la opinión mayoritaria de sus usuarios, aportando mucha fiabilidad.

\section{Objetivos}
En el caso de \textit{My Many Reads}, el objetivo es construir una plataforma social en la que crear una comunidad de usuarios alrededor de la lectura. Partiendo de una base de datos de libros, se pretende reunir a una comunidad de usuarios con intereses en la lectura, ofreciéndoles una serie de servicios y funcionalidades como pueden ser:

\begin{itemize}
    \item Gestión de una biblioteca personal, en la que los usuarios puedan almacenar los libros leídos, pendientes y abandonados junto a una valoración.
    \item Sistema de recomendaciones automático construyendo los gustos de los usuarios a partir de sus bibliotecas personales.
    \item Sistema de búsqueda de libros por popularidad en la plataforma o sus características propias como pueden ser géneros, idioma o fecha de publicación.
    \item Foro de discusiones para cada libro en el que plasmar opiniones, teorías u otros pensamientos relacionados con dicho libro.
    \item Posibilidad de escribir recomendaciones personalizadas entre dos libros similares, que puedan ayudar a otros usuarios a tomar una decisión con su próxima lectura.
\end{itemize}

La meta global por tanto es la de construir una plataforma con características atractivas para atraer a un conjunto de usuarios. Sabiendo esto, el siguiente paso es establecer una lista de objetivos con el fin de planificar y diseñar en torno al cumplimiento de estos, enfocando todo el desarrollo en una dirección. En el caso de \textit{My Many Reads}, estos son los objetivos:

\begin{itemize}
    \item Cumplir con una funcionalidad mínima de consulta de libros y gestión listas o bibliotecas.
    \item Con el fin de que los usuarios puedan encontrar fácilmente buenas sugerencias para sus próximas lecturas, construir un sistema de recomendaciones que, a partir de los gustos del usuario, devuelva una lista de libros que podrían ser de interés para éste.
    \item Permitir a los usuarios participar activamente en creación de contenido de la plataforma a través de recomendaciones entre libros personalizadas, reviews y enlaces a foros de discusiones.
    \item Servir como medio de comunicación para ofrecer noticias pertenecientes al mundo de la lectura.
    \item Recoger estadísticas de los usuarios de la plataforma para conocer información sobre, por ejemplo, las tendencias actuales.
\end{itemize}

Estos dos últimos objetivos tienen cierta dependencia a que el proyecto se encuentre desplegado y con una base mínima de usuarios. Para llegar a ese punto, es imprescindible cumplir los objetivos previos cuya intención es atraer usuarios a la plataforma.
