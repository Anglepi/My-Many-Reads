\chapter{Planificación}

En esta sección se recoge todo lo necesario para comprender cómo se planifica el desarrollo de este proyecto, describiendo las diferentes unidades de trabajo y sus características, enumerando las diferentes tareas planificadas y especificando la organización temporal del proyecto.

Así mismo, se explica la metodología seguida ya que de ésta depende, en parte, la planificación elaborada.

\section{Metodología utilizada}

Se pretende seguir una metodología de desarrollo ágil, elaborando tareas de pequeño tamaño y agrupándolas para cumplir en conjunto un objetivo de relativa importancia. Estas tareas se asignan a un \textit{sprint} de una duración concreta, permitiendo así tener claramente determinadas unidades de trabajo pequeñas y concretas, conformando funcionalidades u objetivos más complejos que se proponen cumplimentar en un intervalo de tiempo determinado.

Pero como en todo, pueden darse imprevistos durante el desarrollo que requieran de un tiempo adicional, por lo que esta planificación no es final en ningún momento, y siempre se debe contemplar la posibilidad de encontrarse con obstáculos no planificados que afecten al desarrollo correspondiente a un \textit{sprint}.

Para tener en cuenta apropiadamente estos imprevistos, como ya se ha mencionado anteriormente, no se crean y planifican tareas inmutables. A mayor avance en el tiempo más grande es la incertidumbre, por lo que se proponen una serie de objetivos poco detallados a cumplir en un largo periodo de tiempo, pero las tareas concretas para alcanzar estos objetivos solo se planifican en el momento en el que se vayan a realizar.

Como ya se comentará más adelante, el proyecto se está desarrollando en GitHub, que ofrece herramientas que permiten elaborar una planificación, como \textit{issues} y \textit{milestones}, e incluso ofrece un tablero de KanBan y un \textit{backlog} de tareas en la sección de proyectos.

\subsection{Issues}

Las issues recogen el motivo u objetivo de un conjunto de cambios, tareas o acciones a realizar para contribuir al desarrollo o mantenimiento del proyecto. Por sí solas no ofrecen una gran versatilidad, pero gracias al sistema de etiquetas personalizables, a proyectos (para el tablero de KanBan y el backlog) y milestones empleadas en conjunto, se consigue un buen entorno de planificación.

La intención es que todos los cambios realizados en el proyecto (salvo quizás alguna excepción puntual) vayan asignados a un \textit{issue}, y que cada \textit{issue} esté asignado a un \textit{milestone} y/o haga referencia a otros issues en caso de considerar necesario subdividir en tareas aún más pequeñas.

Los tipos de \textit{issues} más destacables de mencionar son las historias de usuario (etiqueta \textit{user-story}). Éstas definen las historias de usuario, y por lo general están divididas en tareas, empleando otros issues que hagan referencia a la historia de usuario a la que van a contribuir.

\subsection{Milestones}

Cada \textit{milestone} representa un \textit{Producto Mínimamente Viable} o PMV. Un \textit{milestone} recoge todo el conjunto de tareas (\textit{issues}) necesarios de cumplimentar para alcanzar los requisitos propuestos del PMV propuesto. Del uso de estos \textit{milestones} como representaciones de PMVs se deduce fácilmente que se sigue una metodología que emplea iteraciones para diferentes entregas del proyecto en función de su desarrollo, es decir, cada nuevo \textit{milestone} se construye como un incremento del anterior, por lo que cada PMV debe suponer un incremento notable en la funcionalidad.

El hecho de que estos \textit{milestones} estén compuestos del conjunto de \textit{issues} necesarios para su terminación no quiere decir que absolutamente todos los \textit{issues} del proyecto estén asignados a un \textit{milestone}, puesto que como se ha explicado en la sección anterior, un \textit{issue} es una herramienta muy versátil, permitiendo instancias de éstos dedicadas a tareas cotidianas o de menor importancia que no encajen como parte de un \textit{milestone}, como por ejemplo la corrección de erratas.

Por lo general, un \textit{milestone} estará compuesto de historias de usuario o historias técnicas, pero eso no quiere decir que éstas (mayormente las historias de usuario) puedan formar parte de varios \textit{milestones}. Un ejemplo genérico de este caso sería el de la siguiente historia de usuario:

\begin{center}
    \textit{Como Usuario de My Many Reads,}\\
    \textit{quiero poder consultar recomendaciones personalizadas de libros,}\\
    \textit{para así ayudarme en la toma de decisión de mi próxima lectura.}
\end{center}

Esta historia se podría cumplimentar fácilmente con un algoritmo muy básico de recomendaciones en base a la lista de libros leídos por el usuario. Sin embargo, en el caso de \textit{My Many Reads}, esta historia necesita algo más que un algoritmo básico. Añadir complejidad al algoritmo para producir mejores resultados u ofrecer métodos alternativos de obtener recomendaciones también pueden ser parte de esta historia propuesta, pero no tendrían por qué serlo del mismo PMV, por lo que esta historia podría tener varios niveles de completitud, pudiendo pertenecer algunos de ellos a \textit{milestones} diferentes.

\section{Temporización}

Dado que el proyecto se está enfocando de manera ágil, no procede realizar una temporización completa de éste. Dicha temporización se elabora en varias ocasiones a lo largo del desarrollo, para periodos más cortos de tiempo, pero siempre con objetivos más difusos y abiertos al cambio en largos plazos de tiempo.

En primer lugar hay que categorizar los \textit{issues} de desarrollo del proyecto, para diferenciar entre varios tipos y asignar un tiempo a cada uno de éstos:

\begin{itemize}
    \item \textbf{Epopeyas:} representan un cambio notable en el proyecto. Son una agrupación de tareas, que se completa cuando la última de éstas se ha completado. Cada epopeya debe estar estimada a poder finalizarse en un periodo entre 5 y 8 días hábiles.
    \item \textbf{Tareas:} las unidades mínimas de trabajo. Por sí solas, pueden incluso no suponer ningún cambio en el proyecto. Se deben planificar para ser realizables en 1 a 3 días hábiles.
\end{itemize}

Con varias de estas epopeyas se definiría un \textit{milestone} que, como se indica en el apartado anterior, representa un PMV. Lo ideal sería que cada PMV se desarrollase en entre 4 y 6 semanas, lo que significa aproximadamente 4 epopeyas por PMV, pero es posible que en algunos casos se requerirá más o menos tiempo o epopeyas para completar un \textit{milestone}.

Cabe mencionar que también se emplearán issues que sean excepciones a estas categorías. Éstas pueden consistir en pequeños arreglos independientes, sin epopeya asignada, u objetivos de mayor duración que no encajan con la definición de epopeya pero que tampoco llegan a representar un PMV, quizás pudiendo definirse como \textit{fase}. Existirán pocas excepciones que no encajen en las categorías definidas previamente, y es por eso por lo que se prefiere simplemente dejarlas como excepciones y no intentar categorizar todas ellas.

\section{Control de calidad del código}

Una de las formas de mejorar la calidad del código es la de emplear tests que aseguren el buen funcionamiento del código. Por ello, para cada incremento o cambio en el código, se establece el requisito de asegurar su cobertura de tests, comprobando cada línea de código y las diferentes ramificaciones del mismo (diferentes casos con diferentes condiciones).

Principalmente existen dos alternativas no excluyentes para programar estos tests: los tests unitarios, que comprueban componentes como clases o funciones de forma individual, y los tests de integración, en los que se pone a prueba la interacción de diferentes componentes. Dado que los tests de integración son, generalmente, más complejos que los unitarios, no se va a exigir una cobertura total de todas las interacciones de componentes, es decir, no van a ser obligatorios. Sin embargo, pueden existir casos en los que los test unitarios por sí solos realmente no estén aportando mucha utilidad, por lo que emplear tests de integración para verificar el correcto funcionamiento de los componentes será necesario.

La cobertura objetivo que se propone alcanzar es por encima del 98\%, dado que pueden existir casos muy difíciles de cubrir, que representen casos improbables o que dependan del entorno o la configuración de ejecución. Pero el objetivo no será solamente alcanzar la máxima cobertura posible. Aunque la cobertura de un componente sea del 100\%, es necesario comprobar los diferentes casos posibles, con diferentes datos de entrada y bajo diferentes condiciones. Adicionalmente, siempre que se encuentre un \textit{bug}, habrá que incluir los tests necesarios que se aseguren de que dicho \textit{bug} no vuelva a surgir.

Por último, existe una excepción de modificación en el código que no implica añadir o actualizar los tests. Es el caso de las refactorizaciones, si realmente se trata de estas, los tests deben producir exactamente los mismos resultados al poner a prueba los mismos comportamientos. Un libro muy interesante sobre el tema de la refactorización es el titulado ``\textit{Refactoring}'' de \textit{Martin Fowler}.

\section{Seguimiento del desarrollo}

En la página del repositorio en GitHub se encuentran todas las \href{https://github.com/Anglepi/My-Many-Reads/issues}{\textit{issues}} y \href{https://github.com/Anglepi/My-Many-Reads/milestones}{\textit{milestones}} del proyecto, organizadas como ya se ha descrito anteriormente. Así mismo, en la sección de proyectos, se puede consultar un tablero \href{https://github.com/users/Anglepi/projects/1}{\textit{KanBan}} que contendrá tanto tareas como epopeyas de desarrollo y su estado actual, con el fin de apoyar la organización del desarrollo.

Dado que se están empleando los \href{https://github.com/Anglepi/My-Many-Reads/pulls}{\textit{Pull Requests}} para realizar los incrementos de progreso en el proyecto, éstos también son una fuente muy útil de información acerca del seguimiento y estado del desarrollo. Éstos son un buen lugar para encontrar la documentación del proyecto en un momento dado, ya que entre las tareas de integración continua \href{https://github.com/Anglepi/My-Many-Reads/actions/workflows/latex-build.yml}{hay una que genera este PDF}, tan solo selecciona el \textit{workflow} deseado y lo encontrarás abajo.

Adicionalmente, en la sección de \textit{Implementación} se enumerarán todos los PMV junto con sus epopeyas y las tareas que las forman, así como el tiempo dedicado a cada uno de estos elementos, que se irán actualizando conforme se desarrolle el proyecto.