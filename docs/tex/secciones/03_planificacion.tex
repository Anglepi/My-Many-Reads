\chapter{Planificación}

En esta sección se recoge todo lo necesario para comprender cómo se planifica el desarrollo de este proyecto, describiendo las diferentes unidades de trabajo y sus características, enumerando las diferentes tareas planificadas y especificando la organización temporal del proyecto.
\\\\
Así mismo, se explica la metodología seguida ya que de ésta depende, en parte, la planificación elaborada.

\section{Metodología utilizada}

Se pretende seguir una metodología de desarrollo ágil, elaborando tareas de pequeño tamaño y agrupándolas para cumplir en conjunto un objetivo de relativa importancia. Estas tareas se asignan a un \textit{sprint} de una duración concreta, permitiendo así tener claramente determinadas unidades de trabajo pequeñas y concretas, conformando funcionalidades u objetivos más complejos que se proponen cumplimentar en un intervalo de tiempo determinado.
\\\\
Pero como en todo, pueden darse imprevistos durante el desarrollo que requieran de un tiempo adicional, por lo que esta planificación no es final en ningún momento, y siempre se debe contemplar la posibilidad de encontrarse con obstáculos no planificados que afecten al desarrollo correspondiente a un \textit{sprint}.
\\\\
Para tener en cuenta apropiadamente estos imprevistos, como ya se ha mencionado anteriormente, no se crean y planifican tareas inmutables. A mayor avance en el tiempo más grande es la incertidumbre, por lo que se proponen una serie de objetivos poco detallados a cumplir en un largo periodo de tiempo, pero las tareas concretas para alcanzar estos objetivos solo se planifican en el momento en el que se vayan a realizar.
\\\\
Como ya se comentará más adelante, el proyecto se está desarrollando en GitHub\cite{Github}, que ofrece herramientas que permiten elaborar una planificación, como \textit{issues} y \textit{milestones}, e incluso ofrece un tablero de KanBan y un \textit{backlog} de tareas en la sección de proyectos.

\subsection{Issues}

\subsection{Milestones}

\section{Temporización}

\section{Seguimiento del desarrollo}
