\chapter{Planificación}

En esta sección se recoge todo lo necesario para comprender cómo se planifica el desarrollo de este proyecto, describiendo las diferentes unidades de trabajo y sus características, enumerando las diferentes tareas planificadas y especificando la organización temporal del proyecto.

Así mismo, se explica la metodología seguida ya que de ésta depende, en parte, la planificación elaborada.

\section{Metodología utilizada}

Se pretende seguir una metodología de desarrollo ágil, elaborando tareas de pequeño tamaño y agrupándolas para cumplir en conjunto un objetivo de relativa importancia. Estas tareas se asignan a un \textit{sprint} de una duración concreta, permitiendo así tener claramente determinadas unidades de trabajo pequeñas y concretas, conformando funcionalidades u objetivos más complejos que se proponen cumplimentar en un intervalo de tiempo determinado.

Pero como en todo, pueden darse imprevistos durante el desarrollo que requieran de un tiempo adicional, por lo que esta planificación no es final en ningún momento, y siempre se debe contemplar la posibilidad de encontrarse con obstáculos no planificados que afecten al desarrollo correspondiente a un \textit{sprint}.

Para tener en cuenta apropiadamente estos imprevistos, como ya se ha mencionado anteriormente, no se crean y planifican tareas inmutables. A mayor avance en el tiempo más grande es la incertidumbre, por lo que se proponen una serie de objetivos poco detallados a cumplir en un largo periodo de tiempo, pero las tareas concretas para alcanzar estos objetivos solo se planifican en el momento en el que se vayan a realizar.

Como ya se comentará más adelante, el proyecto se está desarrollando en GitHub\cite{Github}, que ofrece herramientas que permiten elaborar una planificación, como \textit{issues} y \textit{milestones}, e incluso ofrece un tablero de KanBan y un \textit{backlog} de tareas en la sección de proyectos.

\subsection{Issues}

Las issues recogen el motivo u objetivo de un conjunto de cambios, tareas o acciones a realizar para contribuir al desarrollo o mantenimiento del proyecto. Por sí solas no ofrecen una gran versatilidad, pero gracias al sistema de etiquetas personalizables, a proyectos (para el tablero de KanBan y el backlog) y milestones empleadas en conjunto, se consigue un buen entorno de planificación.

La intención es que todos los cambios realizados en el proyecto (salvo quizás alguna excepción puntual) vayan asignados a un \textit{issue}, y que cada \textit{issue} esté asignado a un \textit{milestone} y/o haga referencia a otros issues en caso de considerar necesario subdividir en tareas aún más pequeñas.

Los tipos de \textit{issues} más destacables de mencionar son las historias de usuario (etiqueta \textit{user-story}). Éstas definen las historias de usuario, y por lo general están divididas en tareas, empleando otros issues que hagan referencia a la historia de usuario a la que van a contribuir.

\subsection{Milestones}

Cada \textit{milestone} representa un \textit{Producto Mínimamente Viable} o PMV. Un \textit{milestone} recoge todo el conjunto de tareas (\textit{issues}) necesarios de cumplimentar para alcanzar los requisitos propuestos del PMV propuesto. Del uso de estos \textit{milestones} como representaciones de PMVs se deduce fácilmente que se sigue una metodología que emplea iteraciones para diferentes entregas del proyecto en función de su desarrollo, es decir, cada nuevo \textit{milestone} se construye como un incremento del anterior, por lo que cada PMV debe suponer un incremento notable en la funcionalidad.

El hecho de que estos \textit{milestones} estén compuestos del conjunto de \textit{issues} necesarios para su terminación no quiere decir que absolutamente todos los \textit{issues} del proyecto estén asignados a un \textit{milestone}, puesto que como se ha explicado en la sección anterior, un \textit{issue} es una herramienta muy versátil, permitiendo instancias de éstos dedicadas a tareas cotidianas o de menor importancia que no encajen como parte de un \textit{milestone}, como por ejemplo la corrección de erratas.

Por lo general, un \textit{milestone} estará compuesto de historias de usuario o historias técnicas, pero eso no quiere decir que éstas (mayormente las historias de usuario) puedan formar parte de varios \textit{milestones}. Un ejemplo genérico de este caso sería el de la siguiente historia de usuario:

\begin{center}
    \textit{Como Usuario de My Many Reads,}\\
    \textit{quiero poder consultar recomendaciones personalizadas de libros,}\\
    \textit{para así ayudarme en la toma de decisión de mi próxima lectura.}
\end{center}

Esta historia se podría cumplimentar fácilmente con un algoritmo muy básico de recomendaciones en base a la lista de libros leídos por el usuario. Sin embargo, en el caso de \textit{My Many Reads}, esta historia necesita algo más que un algoritmo básico. Añadir complejidad al algoritmo para producir mejores resultados u ofrecer métodos alternativos de obtener recomendaciones también pueden ser parte de esta historia propuesta, pero no tendrían por qué serlo del mismo PMV, por lo que esta historia podría tener varios niveles de completitud, pudiendo pertenecer algunos de ellos a \textit{milestones} diferentes.

\section{Temporización}

\section{Seguimiento del desarrollo}
