\chapter{Costes del proyecto}

Anteriormente en la sección de planificación se mostró una descripción de la \hyperref[Temporizacion]{\underline{temporización del proyecto}}. Dado que se sigue una filosofía ágil para el proyecto, al principio de este era inviable plantear una planificación temporal completa.

Es por esto que una vez realizado todo el desarrollo, se puede elaborar una estimación de costes más precisa, tomando como referencia la estimación media de tiempo por unidad de trabajo, una aproximación de salario de un informático y la cantidad de tareas realizadas.

En total, se han desarrollado \textbf{10 epopeyas}, las cuales se intentaron definir para completar su desarrollo en 7 días, pero que finalmente se extendieron algunas de ellas hasta los 9 días, debido principalmente a la falta de experiencia con ciertas tecnologías y algunos problemas imprevistos. Adicionalmente, para cada epopeya, hubo que dedicar una media de 3 días en revisiones

Adicionalmente, surgieron \textbf{5 \textit{bugs}}, tres de los cuales eran de relativa simplicidad y se resolvieron en un día cada uno, mientras que los otros dos implicaron algo más de trabajo, dos días por cada uno.

Existen otras tareas, agrupadas como \textit{Architecture and Quality}, independientes y definidas para mejorar el proceso de desarrollo, como por ejemplo refactorizaciones, y otras para trabajar en integración continua, \textit{CI}. En total se ha trabajado en \textbf{5 tareas de \textit{Architecture and Quality}} y \textbf{3 tareas de \textit{CI}}, con una media aproximada de un día por tarea.

Tomando finalmente una aproximación de 100 días para las epopeyas y otros 12 días para tareas inependientes de desarrollo, con una media de 4 horas por día de trabajo, se llegan a las 452 horas dedicadas al desarrollo, mantenimiento y documentación del código.

Existen también otras tareas no contabilizadas en el total anterior, como la creación, configuración y gestión del repositorio, del entorno local de desarrollo, y otras tareas de documentación e investigación como el análisis del problema, el estado del arte y la definición de planificación a emplear. Teniendo esto en cuenta, el tiempo empleado al desarrollo completo del proyecto hasta este punto puede llegar a una aproximación de 500 horas.